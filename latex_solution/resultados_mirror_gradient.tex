
\documentclass[12pt]{article}
\usepackage[utf8]{inputenc}
\usepackage[portuguese]{babel}
\usepackage{booktabs}
\usepackage{array}
\usepackage{geometry}
\usepackage{amsmath}
\usepackage{amsfonts}
\usepackage{pdflscape}

\geometry{a4paper, margin=2cm}

\title{Resultados dos Problemas de Otimização - Método do Gradiente Espelhado}
\author{Análise Computacional}
\date{\today}

\begin{document}

\maketitle

\section{Problemas de Otimização}

Esta tabela apresenta os problemas de otimização não-linear resolvidos usando o método do Gradiente Espelhado (Mirror Descent) e o número de variáveis de cada problema.

\begin{table}[h!]
\centering
\caption{Problemas de otimização e número de variáveis}
\label{tab:problemas_variáveis}
\begin{tabular}{@{}lc@{}}
\toprule
\textbf{Problema} & \textbf{Número de Variáveis} \\
\midrule
ROSENBROCK & 10 \\
PENALTY & 5 \\
TRIGONOMETRIC & 10 \\
EXTENDED ROSENBROCK & 10 \\
EXTENDED POWELL & 12 \\
QOR & 50 \\
GOR & 50 \\
PSP & 50 \\
TRIDIAGONAL & 10 \\
ENGGVAL1 & 10 \\
LINEAR MINIMUM SURFACE & 9 \\
SQUARE ROOT 1 & 16 \\
SQUARE ROOT 2 & 16 \\
FREUDENTHAL ROTH & 10 \\
SPARSE MATRIX SQRT & 10 \\
ULTS0 & 64 \\
\bottomrule
\end{tabular}
\end{table}

\section{Resultados de Convergência}

Esta tabela apresenta os resultados de convergência para cada problema, incluindo o número de iterações necessárias, o valor mínimo da função objetivo encontrado e a precisão da solução (norma do gradiente).

\begin{table}[h!]
\centering
\caption{Resultados de convergência dos problemas de otimização}
\label{tab:resultados_convergencia}

\small
\begin{tabular}{|l|c|c|c|}
\hline
\textbf{Problema} & \textbf{Iterações} & \textbf{Valor Mínimo} & \textbf{Precisão ($||\nabla f(x^*)||$)} & \textbf{Tempo (s)} \\
\hline
ROSENBROCK & 0 & 0.000000e+00 & 3.999688e-10 & 0.000s \\
\hline
PENALTY & 332 & 9.147106e-02 & 9.737285e-05 & 0.336s \\
\hline
TRIGONOMETRIC & 999 & 1.709066e-04 & 6.868246e-03 & 1.002s \\
\hline
EXTENDED ROSENBROCK & 0 & 0.000000e+00 & 8.943575e-10 & 0.000s \\
\hline
EXTENDED POWELL & 7 & 1.430492e+35 & 1.591747e+27 & 0.106s \\
\hline
QOR & 716 & 1.175472e+03 & 2.819951e-04 & 169.942s \\
\hline
GOR & 999 & 1.381140e+03 & 1.061535e-01 & 209.612s \\
\hline
PSP & 999 & 2.029601e+02 & 1.339710e+01 & 164.278s \\
\hline
TRIDIAGONAL & 713 & 3.499057e-09 & 9.660990e-05 & 0.659s \\
\hline
ENGGVAL1 & 464 & 9.177470e+00 & 9.722345e-05 & 0.445s \\
\hline
LINEAR MINIMUM SURFACE & 999 & 3.469804e+00 & 6.846937e-01 & 15.453s \\
\hline
SQUARE ROOT 1 & 999 & 1.075595e-04 & 7.032637e-03 & 1.780s \\
\hline
SQUARE ROOT 2 & 999 & 4.521564e-04 & 3.954493e-03 & 1.716s \\
\hline
FREUDENTHAL ROTH & 3 & 8.915542e+81 & 0.000000e+00 & 0.069s \\
\hline
SPARSE MATRIX SQRT & 999 & 1.799093e-04 & 8.526912e-03 & 1.451s \\
\hline
ULTS0 & 3 & 5.086273e+28 & 0.000000e+00 & 1.047s \\
\hline
\hline
\end{tabular}
\end{table}

\section{Soluções Encontradas (Primeiras 5 Variáveis)}

Esta tabela apresenta as primeiras 5 variáveis da solução encontrada para cada problema. Para problemas com menos de 5 variáveis, apenas as variáveis disponíveis são mostradas.

\begin{landscape}
\begin{table}[h!]
\centering
\caption{Primeiras 5 variáveis das soluções encontradas}
\label{tab:solucoes_variáveis}
\begin{tabular}{|l|c|c|c|c|c|}
\hline
\textbf{Problema} & \textbf{x₁} & \textbf{x₂} & \textbf{x₃} & \textbf{x₄} & \textbf{x₅} \\
\hline
ROSENBROCK & 1.000000e+00 & 1.000000e+00 & 1.000000e+00 & 1.000000e+00 & 1.000000e+00 \\
\hline
PENALTY & 9.815904e-01 & 9.815905e-01 & 9.815905e-01 & 9.815904e-01 & 9.815906e-01 \\
\hline
TRIGONOMETRIC & 6.055791e-02 & 6.273479e-02 & 6.530700e-02 & 6.847020e-02 & 7.277235e-02 \\
\hline
EXTENDED ROSENBROCK & 1.000000e+00 & 1.000000e+00 & 1.000000e+00 & 1.000000e+00 & 1.000000e+00 \\
\hline
EXTENDED POWELL & 1.764691e+00 & -9.190465e+07 & 1.895674e+08 & 1.336783e+00 & 1.764693e+00 \\
\hline
QOR & 5.923306e-01 & -7.111491e-01 & 6.285711e-02 & -2.651213e+00 & 1.583524e+00 \\
\hline
GOR & -1.789059e+00 & -3.514595e-01 & -3.031726e+00 & -1.043714e-01 & 7.394256e+00 \\
\hline
PSP & 4.999685e+00 & 4.943255e+00 & 5.002957e+00 & 2.903052e+00 & 4.995765e+00 \\
\hline
TRIDIAGONAL & 1.000054e+00 & 5.000361e-01 & 2.500240e-01 & 1.250159e-01 & 6.251044e-02 \\
\hline
ENGGVAL1 & 9.010301e-01 & 5.458806e-01 & 6.512110e-01 & 6.240699e-01 & 6.319669e-01 \\
\hline
LINEAR MINIMUM SURFACE & 3.404830e+00 & 6.562140e+00 & 6.361524e+00 & 6.568676e+00 & 4.510520e+00 \\
\hline
SQUARE ROOT 1 & 8.316581e-01 & -1.267529e-01 & -6.284486e-01 & -5.949608e-01 & -7.664529e-01 \\
\hline
SQUARE ROOT 2 & 7.378219e-01 & -2.573644e-01 & -6.349570e-02 & -6.657820e-01 & -7.189708e-01 \\
\hline
FREUDENTHAL ROTH & -6.970890e-01 & -3.857246e+11 & -3.382124e+11 & -3.381847e+11 & -3.382064e+11 \\
\hline
SPARSE MATRIX SQRT & 8.316120e-01 & -7.320686e-01 & 4.043655e-01 & -2.611206e-01 & -1.323311e-01 \\
\hline
ULTS0 & 1.679770e+06 & -2.082768e+06 & 2.461385e+06 & 1.893953e+06 & 6.565687e+05 \\
\hline
\hline
\end{tabular}
\end{table}
\end{landscape}

\section{Observações}

\begin{itemize}
\item O método do Gradiente Espelhado foi configurado com tolerância de convergência de $10^{-6}$ e parâmetro de passo $\eta = 0.01$.
\item Para problemas que falharam, verifique a mensagem de erro específica.
\item A precisão é medida pela norma do gradiente ($||\nabla f(x^*)||$) calculada numericamente.
\item Valores de precisão menores indicam soluções mais próximas de pontos estacionários.
\item Para problemas irrestritos, $||\nabla f(x^*)|| \approx 0$ indica convergência para um mínimo local.
\item Problemas que falharam são marcados com "---" nas colunas de resultados.
\item A terceira tabela mostra as primeiras 5 variáveis da solução encontrada.
\item Para problemas com menos de 5 variáveis, as colunas extras são marcadas como "---".
\item A terceira tabela é apresentada em formato paisagem para melhor visualização.
\item O método do Gradiente Espelhado usa divergência de Bregman com função euclidiana $\phi(x) = \frac{1}{2}||x||^2$.
\end{itemize}

\end{document}
