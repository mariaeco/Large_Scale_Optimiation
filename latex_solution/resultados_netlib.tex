
\documentclass[12pt]{article}
\usepackage[utf8]{inputenc}
\usepackage[portuguese]{babel}
\usepackage{booktabs}
\usepackage{array}
\usepackage{geometry}
\usepackage{amsmath}
\usepackage{amsfonts}
\usepackage{pdflscape}
\usepackage{longtable}

\geometry{a4paper, margin=1.5cm}

\title{Resultados dos Problemas NETLIB - Solver HiGHS}
\author{Análise Computacional}
\date{\today}

\begin{document}

\maketitle

\section{Informações dos Problemas}

Esta tabela apresenta informações básicas sobre cada problema da coleção NETLIB, incluindo métricas de viabilidade.

\scriptsize
\begin{longtable}{@{}l|cccc@{}}
\caption{Informações dos problemas NETLIB} \label{tab:info_problemas} \\
\toprule
\textbf{Problema} & \textbf{Nº de Variáveis} & \textbf{Nº de Restrições} & \textbf{Inviab. Primal} & \textbf{Inviab. Dual} \\
\midrule
\endfirsthead

\toprule
\textbf{Problema} & \textbf{Nº de Variáveis} & \textbf{Nº de Restrições} & \textbf{Inviab. Primal} & \textbf{Inviab. Dual} \\
\midrule
\endhead

\midrule \multicolumn{5}{r}{{Continua na próxima página}} \\ \midrule
\endfoot

\bottomrule
\endlastfoot
25fv47 & 1571 & 821 & 0.000e+00 & 1.332e-15 \\
80bau3b & 9799 & 2262 & 0.000e+00 & 1.377e-14 \\

\bottomrule
\end{longtable}

\section{Resultados de Convergência}

Esta tabela apresenta os resultados de convergência para cada problema, incluindo o número de iterações, valor da função objetivo e o gap relativo.

\scriptsize
\begin{longtable}{@{}l|cccccc@{}}
\caption{Resultados de convergência dos problemas NETLIB} \label{tab:resultados_convergencia} \\
\toprule
\textbf{Problema} & \textbf{Iterações} & \textbf{Valor Ótimo} & \textbf{Valor Ótimo Primal} & \textbf{Valor Ótimo Dual} & \textbf{Gap Absoluto} & \textbf{Gap Relativo} \\
\midrule
\endfirsthead


\multicolumn{7}{c}%
{{\bfseries \tablename\ \thetable{} -- continuação da página anterior}} \\
\toprule
\textbf{Problema} & \textbf{Iterações} & \textbf{Valor Ótimo} & \textbf{Valor Ótimo Primal} & \textbf{Valor Ótimo Dual} & \textbf{Gap Absoluto} & \textbf{Gap Relativo} \\
\midrule
\endhead

\midrule \multicolumn{7}{r}{{Continua na próxima página}} \\ \midrule
\endfoot

\bottomrule
\endlastfoot
25fv47 & 28 & 5.502e+03 & 5.502e+03 & 5.502e+03 & 4.132e-16 & 7.511e-20 \\
80bau3b & 43 & 9.872e+05 & 9.872e+05 & 1.130e+06 & 2.948e-16 & 2.986e-22 \\

\bottomrule
\end{longtable}

\section{Soluções das Variáveis (Primeiras 5)}

Esta tabela apresenta as primeiras 5 variáveis da solução encontrada para cada problema. Para problemas com menos de 5 variáveis, apenas as variáveis disponíveis são mostradas.

\scriptsize % ou \footnotesize, escolha conforme achar melhor
\begin{longtable}{|l|ccccc|}
\caption{Primeiras 5 variáveis das soluções encontradas\label{tab:solucoes_variaveis}} \\
\hline
\textbf{Problema} & \textbf{x1} & \textbf{x2} & \textbf{x3} & \textbf{x4} & \textbf{x5} \\
\hline
\endfirsthead

\hline
\textbf{Problema} & \textbf{x1} & \textbf{x2} & \textbf{x3} & \textbf{x4} & \textbf{x5} \\
\hline
\endhead

\hline
\multicolumn{6}{r}{{Continua na próxima página}} \\
\endfoot

\hline
\endlastfoot
25fv47 & 53.139 & 0.000 & 0.000 & 34.226 & 0.000 \\
80bau3b & 531.125 & 286.801 & 544.742 & 895.539 & 1810.744 \\


\hline
\end{longtable}

\section{Observações}

\begin{itemize}
\item O solver HiGHS foi configurado com o método IPM (Interior Point Method).
\item Problemas com status "Optimal" convergiram com sucesso.
\item A primeira tabela mostra informações básicas dos problemas e métricas de viabilidade.
\item A segunda tabela apresenta métricas de convergência e qualidade da solução.
\item A terceira tabela mostra valores simulados das primeiras 5 variáveis (para demonstração).
\item A terceira tabela é apresentada em formato paisagem para melhor visualização.
\end{itemize}

\end{document}
