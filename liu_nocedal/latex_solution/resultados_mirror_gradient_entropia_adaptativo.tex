
\documentclass[12pt]{article}
\usepackage[utf8]{inputenc}
\usepackage[portuguese]{babel}
\usepackage{booktabs}
\usepackage{array}
\usepackage{geometry}
\usepackage{amsmath}
\usepackage{amsfonts}
\usepackage{pdflscape}

\geometry{a4paper, margin=2cm}

\title{Resultados dos Problemas de Otimização - Método Gradiente Espelhado Otimizado (Entropia negativa (adaptativo))}
\author{Análise Computacional}
\date{\today}

\begin{document}

\maketitle

\section{Problemas de Otimização}

A tabela 1 apresenta os problemas de otimização não-linear resolvidos usando o método Gradiente Espelhado Otimizado (Entropia negativa (adaptativo)) e o número de variáveis de cada problema.

\begin{table}[h!]
\centering
\caption{Problemas de otimização e número de variáveis}
\label{tab:problemas_variáveis}
\begin{tabular}{@{}|lc|@{}}
\toprule
\textbf{Problema} & \textbf{Número de Variáveis} \\
\midrule
ROSENBROCK & 100 \\
PENALTY & 100 \\
TRIGONOMETRIC & 100 \\
EXTENDED ROSENBROCK & 100 \\
EXTENDED POWELL & 100 \\
QOR & 50 \\
GOR & 50 \\
PSP & 50 \\
TRIDIAGONAL & 100 \\
ENGGVAL1 & 100 \\
LINEAR MINIMUM SURFACE & 36 \\
SQUARE ROOT 1 & 36 \\
SQUARE ROOT 2 & 36 \\
FREUDENTHAL ROTH & 100 \\
SPARSE MATRIX SQRT & 16 \\
ULTS0 & 64 \\
\bottomrule
\end{tabular}
\end{table}

\section{Resultados de Convergência}

A tabela 2 apresenta os resultados de convergência para cada problema, incluindo o número de iterações necessárias, o valor mínimo da função objetivo encontrado e a precisão da solução (norma do gradiente).


\begin{table}[h!]
\small
\centering
\caption{Resultados de convergência dos problemas de otimização}
\label{tab:resultados_convergencia}
\small
\begin{tabular}{|l|cccc|}
\hline
\textbf{Problema} & \textbf{Iterações} & \textbf{Valor Mínimo} & \textbf{Precisão ($||\nabla f(x^*)||$)} & \textbf{Tempo (s)}\\
\hline
ROSENBROCK & 1 & 1.000e+00 & 2.000e+00 & 0.003s \\
PENALTY & 1 & 1.000e+02 & 2.000e+01 & 0.009s \\
TRIGONOMETRIC & 1 & 9.797e-05 & 1.940e-02 & 0.000s \\
EXTENDED ROSENBROCK & 1 & 5.000e+01 & 1.414e+01 & 0.008s \\
EXTENDED POWELL & 1 & 0.000e+00 & 0.000e+00 & 0.014s \\
QOR & 1 & 2.335e+03 & 2.062e+02 & 0.012s \\
GOR & 1 & 5.074e+03 & 5.960e+02 & 0.009s \\
PSP & 1 & 1.828e+03 & 1.085e+02 & 0.011s \\
TRIDIAGONAL & 1 & 9.497e+01 & 1.985e+01 & 0.011s \\
ENGGVAL1 & 1 & 2.970e+02 & 3.980e+01 & 0.019s \\
LINEAR MINIMUM SURFACE & 1 & 2.135e+01 & 1.110e+00 & 0.005s \\
SQUARE ROOT 1 & 500 & 2.914e+01 & 6.618e+00 & 0.435s \\
SQUARE ROOT 2 & 500 & 2.894e+01 & 6.842e+00 & 0.482s \\
FREUDENTHAL ROTH & 1 & 9.999e+04 & 7.770e+03 & 0.032s \\
SPARSE MATRIX SQRT & 500 & 1.391e+01 & 4.117e+00 & 0.616s \\
ULTS0 & 44 & 6.822e+04 & 3.650e+05 & 0.534s \\
\hline
\end{tabular}
\end{table}


\section{Soluções Encontradas (Primeiras 5 Variáveis)}

A tabela 3 apresenta as primeiras 5 variáveis da solução encontrada para cada problema. Para problemas com menos de 5 variáveis, apenas as variáveis disponíveis são mostradas.

\begin{landscape}
\begin{table}[h!]
\centering
\caption{Primeiras 5 variáveis das soluções encontradas}
\label{tab:solucoes_variáveis}
\begin{tabular}{|l|ccccc|}
\hline
\textbf{Problema} & \textbf{x1} & \textbf{x2} & \textbf{x3} & \textbf{x4} & \textbf{x5} \\
\hline
ROSENBROCK & 0.000000e+00 & 0.000000e+00 & 0.000000e+00 & 0.000000e+00 & 0.000000e+00 \\
PENALTY & 1.020201e-10 & 1.020201e-10 & 1.020201e-10 & 1.020201e-10 & 1.020201e-10 \\
TRIGONOMETRIC & 9.998060e-03 & --- & --- & --- & --- \\
EXTENDED ROSENBROCK & 1.020201e-10 & 1.000000e-10 & 1.020201e-10 & 1.000000e-10 & 1.020201e-10 \\
EXTENDED POWELL & 0.000000e+00 & 0.000000e+00 & 0.000000e+00 & 0.000000e+00 & 0.000000e+00 \\
QOR & 1.051271e-10 & 9.512294e-11 & 1.476981e-10 & 9.093729e-11 & 1.552707e-10 \\
GOR & 8.954568e-11 & 3.174443e-10 & 8.083234e-11 & 3.545054e-10 & 1.520983e-10 \\
PSP & 1.133375e-10 & 1.150538e-10 & 1.271554e-10 & 1.150308e-10 & 1.190941e-10 \\
TRIDIAGONAL & 1.020201e+00 & 9.801987e-01 & 9.801987e-01 & 9.801987e-01 & 9.801987e-01 \\
ENGGVAL1 & 1.040811e-10 & 1.040811e-10 & 1.040811e-10 & 1.040811e-10 & 1.040811e-10 \\
LINEAR MINIMUM SURFACE & 9.989223e-01 & 2.595828e+00 & 4.191486e+00 & 5.788316e+00 & 7.394464e+00 \\
SQUARE ROOT 1 & 1.682942e-01 & -1.513605e-01 & 8.242370e-02 & -5.758066e-02 & -2.647035e-02 \\
SQUARE ROOT 2 & 1.682942e-01 & -1.513605e-01 & 8.242370e-02 & -5.758066e-02 & -2.647035e-02 \\
FREUDENTHAL ROTH & 2.316367e-10 & 4.097350e-14 & 4.097350e-14 & 4.097350e-14 & 4.097350e-14 \\
SPARSE MATRIX SQRT & 1.682942e-01 & -1.513605e-01 & 8.242370e-02 & -5.758066e-02 & -2.647035e-02 \\
ULTS0 & 3.012102e-02 & 3.978465e-09 & 2.979092e-06 & 4.524322e-04 & 7.366002e-11 \\
\hline
\hline
\end{tabular}
\end{table}
\end{landscape}

\section{Observações}

\begin{itemize}
\item O método L-BFGS-B foi configurado com tolerância de convergência de $10^{-6}$.
\item Para problemas que falharam, verifique a mensagem de erro específica.
\item A precisão é medida pela norma do gradiente ($||\nabla f(x^*)||$) calculada numericamente.
\item Valores de precisão menores indicam soluções mais próximas de pontos estacionários.
\item Para problemas irrestritos, $||\nabla f(x^*)|| \approx 0$ indica convergência para um mínimo local.
\item Problemas que falharam são marcados com "---" nas colunas de resultados.
\item A terceira tabela mostra as primeiras 5 variáveis da solução encontrada.
\item Para problemas com menos de 5 variáveis, as colunas extras são marcadas como "---".
\item A terceira tabela é apresentada em formato paisagem para melhor visualização.
\end{itemize}

\end{document}
