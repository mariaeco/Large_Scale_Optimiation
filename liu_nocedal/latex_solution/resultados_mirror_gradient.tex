
\documentclass[12pt]{article}
\usepackage[utf8]{inputenc}
\usepackage[portuguese]{babel}
\usepackage{booktabs}
\usepackage{array}
\usepackage{geometry}
\usepackage{amsmath}
\usepackage{amsfonts}
\usepackage{pdflscape}

\geometry{a4paper, margin=2cm}

\title{Resultados dos Problemas de Otimização - Método Gradiente Espelhado}
\author{Análise Computacional}
\date{\today}

\begin{document}

\maketitle

\section{Problemas de Otimização}

A tabela 1 apresenta os problemas de otimização não-linear resolvidos usando o método Gradiente Espelhado e o número de variáveis de cada problema.

\begin{table}[h!]
\centering
\caption{Problemas de otimização e número de variáveis}
\label{tab:problemas_variáveis}
\begin{tabular}{@{}|lc|@{}}
\toprule
\textbf{Problema} & \textbf{Número de Variáveis} \\
\midrule
ROSENBROCK & 100 \\
PENALTY & 100 \\
TRIGONOMETRIC & 100 \\
EXTENDED ROSENBROCK & 100 \\
EXTENDED POWELL & 100 \\
QOR & 50 \\
GOR & 50 \\
PSP & 50 \\
TRIDIAGONAL & 100 \\
ENGGVAL1 & 100 \\
LINEAR MINIMUM SURFACE & 36 \\
SQUARE ROOT 1 & 36 \\
SQUARE ROOT 2 & 36 \\
FREUDENTHAL ROTH & 100 \\
SPARSE MATRIX SQRT & 16 \\
ULTS0 & 64 \\
\bottomrule
\end{tabular}
\end{table}

\section{Resultados de Convergência}

A tabela 2 apresenta os resultados de convergência para cada problema, incluindo o número de iterações necessárias, o valor mínimo da função objetivo encontrado e a precisão da solução (norma do gradiente).


\begin{table}[h!]
\small
\centering
\caption{Resultados de convergência dos problemas de otimização}
\label{tab:resultados_convergencia}
\small
\begin{tabular}{|l|cccc|}
\hline
\textbf{Problema} & \textbf{Iterações} & \textbf{Valor Mínimo} & \textbf{Precisão ($||\nabla f(x^*)||$)} & \textbf{Tempo (s)}\\
\hline
ROSENBROCK & 38 & 1.166e+28 & 1.418e+22 & 1.027s \\
PENALTY & 199 & 7.383e+00 & 9.651e-02 & 6.643s \\
TRIGONOMETRIC & 1 & 9.428e-05 & 1.904e-02 & 0.000s \\
EXTENDED ROSENBROCK & 39 & 1.427e+95 & 0.000e+00 & 1.904s \\
EXTENDED POWELL & 1 & 0.000e+00 & 0.000e+00 & 0.001s \\
QOR & 305 & 1.175e+03 & 9.740e-02 & 4.591s \\
GOR & 1000 & 1.381e+03 & 1.063e-01 & 20.323s \\
PSP & 1000 & 2.030e+02 & 1.340e+01 & 15.570s \\
TRIDIAGONAL & 266 & 2.561e-03 & 9.698e-02 & 7.384s \\
ENGGVAL1 & 60 & 1.091e+02 & 8.954e-02 & 2.258s \\
LINEAR MINIMUM SURFACE & 1000 & 1.250e+01 & 8.086e-01 & 7.321s \\
SQUARE ROOT 1 & 458 & 3.639e-02 & 9.977e-02 & 2.745s \\
SQUARE ROOT 2 & 430 & 3.897e-02 & 9.953e-02 & 2.519s \\
FREUDENTHAL ROTH & 4 & 1.595e+112 & 0.000e+00 & 0.666s \\
SPARSE MATRIX SQRT & 234 & 8.687e-03 & 9.820e-02 & 0.627s \\
ULTS0 & 10 & 1.400e+29 & 1.759e+19 & 1.513s \\
\hline
\end{tabular}
\end{table}


\section{Soluções Encontradas (Primeiras 5 Variáveis)}

A tabela 3 apresenta as primeiras 5 variáveis da solução encontrada para cada problema. Para problemas com menos de 5 variáveis, apenas as variáveis disponíveis são mostradas.

\begin{landscape}
\begin{table}[h!]
\centering
\caption{Primeiras 5 variáveis das soluções encontradas}
\label{tab:solucoes_variáveis}
\begin{tabular}{|l|ccccc|}
\hline
\textbf{Problema} & \textbf{x1} & \textbf{x2} & \textbf{x3} & \textbf{x4} & \textbf{x5} \\
\hline
ROSENBROCK & -3.285877e+06 & 1.835208e+04 & -1.680765e-07 & -1.964098e-07 & -1.680765e-07 \\
PENALTY & 8.658028e-01 & 8.658022e-01 & 8.658019e-01 & 8.658019e-01 & 8.658030e-01 \\
TRIGONOMETRIC & 9.805968e-03 & --- & --- & --- & --- \\
EXTENDED ROSENBROCK & -1.358975e+06 & 6.414088e+03 & -1.023360e+06 & 5.288637e+03 & -5.716622e+05 \\
EXTENDED POWELL & 0.000000e+00 & 0.000000e+00 & 0.000000e+00 & 0.000000e+00 & 0.000000e+00 \\
QOR & 5.929098e-01 & -7.098413e-01 & 6.259695e-02 & -2.652335e+00 & 1.587639e+00 \\
GOR & -1.789064e+00 & -3.514722e-01 & -3.031736e+00 & -1.043738e-01 & 7.394266e+00 \\
PSP & 4.999677e+00 & 4.943239e+00 & 5.002947e+00 & 2.903056e+00 & 4.995754e+00 \\
TRIDIAGONAL & 1.022063e+00 & 5.149905e-01 & 2.605432e-01 & 1.328197e-01 & 6.871074e-02 \\
ENGGVAL1 & 8.920491e-01 & 5.543116e-01 & 6.468898e-01 & 6.262327e-01 & 6.306800e-01 \\
LINEAR MINIMUM SURFACE & 1.170689e+00 & 1.551960e+00 & 2.229014e+00 & 3.868261e+00 & 6.582620e+00 \\
SQUARE ROOT 1 & 4.988031e-01 & -4.618385e-01 & -3.944641e-01 & 4.858996e-01 & -5.868193e-01 \\
SQUARE ROOT 2 & 1.613235e-01 & -7.953026e-01 & -5.527543e-01 & 3.474098e-01 & -7.612946e-01 \\
FREUDENTHAL ROTH & -1.437040e+01 & -4.203454e+17 & -4.179566e+17 & -4.188646e+17 & -4.179566e+17 \\
SPARSE MATRIX SQRT & 7.940747e-01 & -5.821523e-01 & 3.895815e-01 & -1.539272e-01 & -1.215970e-01 \\
ULTS0 & 1.471276e+06 & -3.030324e+06 & 2.571035e+06 & 1.094777e+06 & 5.766112e+05 \\
\hline
\hline
\end{tabular}
\end{table}
\end{landscape}

\section{Observações}

\begin{itemize}
\item O método L-BFGS-B foi configurado com tolerância de convergência de $10^{-6}$.
\item Para problemas que falharam, verifique a mensagem de erro específica.
\item A precisão é medida pela norma do gradiente ($||\nabla f(x^*)||$) calculada numericamente.
\item Valores de precisão menores indicam soluções mais próximas de pontos estacionários.
\item Para problemas irrestritos, $||\nabla f(x^*)|| \approx 0$ indica convergência para um mínimo local.
\item Problemas que falharam são marcados com "---" nas colunas de resultados.
\item A terceira tabela mostra as primeiras 5 variáveis da solução encontrada.
\item Para problemas com menos de 5 variáveis, as colunas extras são marcadas como "---".
\item A terceira tabela é apresentada em formato paisagem para melhor visualização.
\end{itemize}

\end{document}
