
\documentclass[12pt]{article}
\usepackage[utf8]{inputenc}
\usepackage[portuguese]{babel}
\usepackage{booktabs}
\usepackage{array}
\usepackage{geometry}
\usepackage{amsmath}
\usepackage{amsfonts}
\usepackage{pdflscape}

\geometry{a4paper, margin=2cm}

\title{Resultados dos Problemas de Otimização - Método L-BFGS-B}
\author{Análise Computacional}
\date{\today}

\begin{document}

\maketitle

\section{Problemas de Otimização}

A tabela 1 apresenta os problemas de otimização não-linear resolvidos usando o método L-BFGS-B e o número de variáveis de cada problema.

\begin{table}[h!]
\centering
\caption{Problemas de otimização e número de variáveis}
\label{tab:problemas_variáveis}
\begin{tabular}{@{}|lc|@{}}
\toprule
\textbf{Problema} & \textbf{Número de Variáveis} \\
\midrule
ROSENBROCK & 100 \\
PENALTY & 100 \\
TRIGONOMETRIC & 100 \\
EXTENDED ROSENBROCK & 100 \\
EXTENDED POWELL & 100 \\
QOR & 100 \\
GOR & 100 \\
PSP & 100 \\
TRIDIAGONAL & 100 \\
ENGGVAL1 & 100 \\
LINEAR MINIMUM SURFACE & 64 \\
SQUARE ROOT 1 & 36 \\
SQUARE ROOT 2 & 36 \\
FREUDENTHAL ROTH & 100 \\
SPARSE MATRIX SQRT & 16 \\
ULTS0 & 64 \\
\bottomrule
\end{tabular}
\end{table}

\section{Resultados de Convergência}

A tabela 2 apresenta os resultados de convergência para cada problema, incluindo o número de iterações necessárias, o valor mínimo da função objetivo encontrado e a precisão da solução (norma do gradiente).


\begin{table}[h!]
\small
\centering
\caption{Resultados de convergência dos problemas de otimização}
\label{tab:resultados_convergencia}
\small
\begin{tabular}{|l|cccc|}
\hline
\textbf{Problema} & \textbf{Iterações} & \textbf{Valor Mínimo} & \textbf{Precisão ($||\nabla f(x^*)||$)} & \textbf{Tempo (s)}\\
\hline
ROSENBROCK & 21 & 1.001e-10 & 2.266e-04 & 0.040s \\
PENALTY & 7 & 7.381e+00 & 5.160e-07 & 0.045s \\
TRIGONOMETRIC & 4 & 5.348e-12 & 4.625e-06 & 0.000s \\
EXTENDED ROSENBROCK & 22 & 5.493e-07 & 2.231e-02 & 0.151s \\
EXTENDED POWELL & 1 & 0.000e+00 & 0.000e+00 & 0.008s \\
QOR & 19 & 1.175e+03 & 9.013e-02 & 0.152s \\
GOR & 47 & 1.381e+03 & 3.632e-01 & 0.504s \\
PSP & 76 & 2.020e+02 & 1.109e-01 & 0.700s \\
TRIDIAGONAL & 17 & 1.564e-07 & 1.109e-03 & 0.106s \\
ENGGVAL1 & 8 & 1.091e+02 & 1.705e-02 & 0.100s \\
LINEAR MINIMUM SURFACE & 37 & 1.000e+00 & 1.021e-03 & 0.251s \\
SQUARE ROOT 1 & 70 & 1.647e-05 & 3.565e-03 & 0.081s \\
SQUARE ROOT 2 & 108 & 2.910e-05 & 4.153e-03 & 0.156s \\
FREUDENTHAL ROTH & 17 & 1.196e+04 & 1.060e+00 & 0.568s \\
SPARSE MATRIX SQRT & 20 & 5.453e-07 & 2.078e-03 & 0.034s \\
ULTS0 & 156 & 1.152e-04 & inf & 1.636s \\
\hline
\end{tabular}
\end{table}


\section{Soluções Encontradas (Primeiras 5 Variáveis)}

A tabela 3 apresenta as primeiras 5 variáveis da solução encontrada para cada problema. Para problemas com menos de 5 variáveis, apenas as variáveis disponíveis são mostradas.

\begin{landscape}
\begin{table}[h!]
\centering
\caption{Primeiras 5 variáveis das soluções encontradas}
\label{tab:solucoes_variáveis}
\begin{tabular}{|l|ccccc|}
\hline
\textbf{Problema} & \textbf{x1} & \textbf{x2} & \textbf{x3} & \textbf{x4} & \textbf{x5} \\
\hline
ROSENBROCK & 9.999916e-01 & 9.999826e-01 & 0.000000e+00 & 0.000000e+00 & 0.000000e+00 \\
PENALTY & 8.691291e-01 & 8.691291e-01 & 8.691291e-01 & 8.691291e-01 & 8.691291e-01 \\
TRIGONOMETRIC & 2.312545e-06 & --- & --- & --- & --- \\
EXTENDED ROSENBROCK & 9.999425e-01 & 9.998785e-01 & 9.999425e-01 & 9.998785e-01 & 9.999425e-01 \\
EXTENDED POWELL & 0.000000e+00 & 0.000000e+00 & 0.000000e+00 & 0.000000e+00 & 0.000000e+00 \\
QOR & 5.905234e-01 & -7.117131e-01 & 5.928532e-02 & -2.650935e+00 & 1.582986e+00 \\
GOR & -1.880962e+00 & -4.128796e-01 & -3.102558e+00 & -1.193911e-01 & 7.336358e+00 \\
PSP & 5.006066e+00 & 4.947234e+00 & 5.002955e+00 & 2.902938e+00 & 4.994337e+00 \\
TRIDIAGONAL & 9.999419e-01 & 4.999728e-01 & 2.499722e-01 & 1.249701e-01 & 6.248340e-02 \\
ENGGVAL1 & 9.018694e-01 & 5.463760e-01 & 6.516042e-01 & 6.240971e-01 & 6.314400e-01 \\
LINEAR MINIMUM SURFACE & 3.062592e+00 & 3.062093e+00 & 3.062467e+00 & 3.062223e+00 & 3.062181e+00 \\
SQUARE ROOT 1 & 6.429353e-01 & -2.241539e-01 & -2.122662e-01 & 4.815893e-01 & -5.907436e-01 \\
SQUARE ROOT 2 & 2.236078e-01 & -6.021510e-01 & -5.360419e-01 & 5.403462e-01 & -8.602652e-01 \\
FREUDENTHAL ROTH & 1.226362e+01 & -8.323555e-01 & -1.507075e+00 & -1.534507e+00 & -1.535822e+00 \\
SPARSE MATRIX SQRT & 8.413714e-01 & -7.560858e-01 & 4.121468e-01 & -2.874876e-01 & -1.324328e-01 \\
ULTS0 & 8.804336e-03 & -1.787901e-03 & -8.848281e-03 & -1.859076e-02 & -2.422501e-02 \\
\hline
\hline
\end{tabular}
\end{table}
\end{landscape}

\section{Observações}

\begin{itemize}
\item O método L-BFGS-B foi configurado com tolerância de convergência de $10^{-6}$.
\item Para problemas que falharam, verifique a mensagem de erro específica.
\item A precisão é medida pela norma do gradiente ($||\nabla f(x^*)||$) calculada numericamente.
\item Valores de precisão menores indicam soluções mais próximas de pontos estacionários.
\item Para problemas irrestritos, $||\nabla f(x^*)|| \approx 0$ indica convergência para um mínimo local.
\item Problemas que falharam são marcados com "---" nas colunas de resultados.
\item A terceira tabela mostra as primeiras 5 variáveis da solução encontrada.
\item Para problemas com menos de 5 variáveis, as colunas extras são marcadas como "---".
\item A terceira tabela é apresentada em formato paisagem para melhor visualização.
\end{itemize}

\end{document}
