
\documentclass[12pt]{article}
\usepackage[utf8]{inputenc}
\usepackage[portuguese]{babel}
\usepackage{booktabs}
\usepackage{array}
\usepackage{geometry}
\usepackage{amsmath}
\usepackage{amsfonts}
\usepackage{pdflscape}

\geometry{a4paper, margin=2cm}

\title{Resultados dos Problemas de Otimização - Método Gradiente Espelhado Otimizado (Norma-p (passo fixo))}
\author{Análise Computacional}
\date{\today}

\begin{document}

\maketitle

\section{Problemas de Otimização}

A tabela 1 apresenta os problemas de otimização não-linear resolvidos usando o método Gradiente Espelhado Otimizado (Norma-p (passo fixo)) e o número de variáveis de cada problema.

\begin{table}[h!]
\centering
\caption{Problemas de otimização e número de variáveis}
\label{tab:problemas_variáveis}
\begin{tabular}{@{}|lc|@{}}
\toprule
\textbf{Problema} & \textbf{Número de Variáveis} \\
\midrule
ROSENBROCK & 100 \\
PENALTY & 100 \\
TRIGONOMETRIC & 100 \\
EXTENDED ROSENBROCK & 100 \\
EXTENDED POWELL & 100 \\
QOR & 50 \\
GOR & 50 \\
PSP & 50 \\
TRIDIAGONAL & 100 \\
ENGGVAL1 & 100 \\
LINEAR MINIMUM SURFACE & 36 \\
SQUARE ROOT 1 & 36 \\
SQUARE ROOT 2 & 36 \\
FREUDENTHAL ROTH & 100 \\
SPARSE MATRIX SQRT & 16 \\
ULTS0 & 64 \\
\bottomrule
\end{tabular}
\end{table}

\section{Resultados de Convergência}

A tabela 2 apresenta os resultados de convergência para cada problema, incluindo o número de iterações necessárias, o valor mínimo da função objetivo encontrado e a precisão da solução (norma do gradiente).


\begin{table}[h!]
\small
\centering
\caption{Resultados de convergência dos problemas de otimização}
\label{tab:resultados_convergencia}
\small
\begin{tabular}{|l|cccc|}
\hline
\textbf{Problema} & \textbf{Iterações} & \textbf{Valor Mínimo} & \textbf{Precisão ($||\nabla f(x^*)||$)} & \textbf{Tempo (s)}\\
\hline
ROSENBROCK & 39 & 1.297e+80 & 0.000e+00 & 0.017s \\
PENALTY & 277 & 7.381e+00 & 9.698e-03 & 2.347s \\
TRIGONOMETRIC & 36 & 2.387e-05 & 9.675e-03 & 0.000s \\
EXTENDED ROSENBROCK & 39 & 6.662e+81 & 0.000e+00 & 0.231s \\
EXTENDED POWELL & 1 & 0.000e+00 & 0.000e+00 & 0.000s \\
QOR & 474 & 1.175e+03 & 9.736e-03 & 2.770s \\
GOR & 500 & 1.381e+03 & 4.245e-01 & 4.563s \\
PSP & 500 & 2.030e+02 & 1.340e+01 & 3.427s \\
TRIDIAGONAL & 395 & 3.077e-05 & 9.782e-03 & 2.894s \\
ENGGVAL1 & 93 & 1.091e+02 & 9.178e-03 & 1.283s \\
LINEAR MINIMUM SURFACE & 500 & 1.649e+01 & 9.463e-01 & 1.201s \\
SQUARE ROOT 1 & 500 & 3.257e-02 & 9.107e-02 & 0.550s \\
SQUARE ROOT 2 & 500 & 3.314e-02 & 8.365e-02 & 0.496s \\
FREUDENTHAL ROTH & 4 & 1.567e+112 & 0.000e+00 & 0.136s \\
SPARSE MATRIX SQRT & 500 & 1.583e-03 & 2.747e-02 & 0.630s \\
ULTS0 & 4 & 1.100e+29 & 0.000e+00 & 0.051s \\
\hline
\end{tabular}
\end{table}


\section{Soluções Encontradas (Primeiras 5 Variáveis)}

A tabela 3 apresenta as primeiras 5 variáveis da solução encontrada para cada problema. Para problemas com menos de 5 variáveis, apenas as variáveis disponíveis são mostradas.

\begin{landscape}
\begin{table}[h!]
\centering
\caption{Primeiras 5 variáveis das soluções encontradas}
\label{tab:solucoes_variáveis}
\begin{tabular}{|l|ccccc|}
\hline
\textbf{Problema} & \textbf{x1} & \textbf{x2} & \textbf{x3} & \textbf{x4} & \textbf{x5} \\
\hline
ROSENBROCK & 3.374401e+19 & 1.275445e+04 & 0.000000e+00 & 0.000000e+00 & 0.000000e+00 \\
PENALTY & 8.687953e-01 & 8.687953e-01 & 8.687953e-01 & 8.687953e-01 & 8.687953e-01 \\
TRIGONOMETRIC & 4.909502e-03 & --- & --- & --- & --- \\
EXTENDED ROSENBROCK & 3.439272e+19 & 1.280708e+04 & 3.439272e+19 & 1.280708e+04 & 3.439272e+19 \\
EXTENDED POWELL & 0.000000e+00 & 0.000000e+00 & 0.000000e+00 & 0.000000e+00 & 0.000000e+00 \\
QOR & 5.924010e-01 & -7.110108e-01 & 6.283535e-02 & -2.651305e+00 & 1.583921e+00 \\
GOR & -1.758101e+00 & -3.722284e-01 & -3.006531e+00 & -9.619996e-02 & 7.535326e+00 \\
PSP & 4.999665e+00 & 4.943251e+00 & 5.002957e+00 & 2.903136e+00 & 4.995767e+00 \\
TRIDIAGONAL & 1.004064e+00 & 5.027260e-01 & 2.518504e-01 & 1.262818e-01 & 6.341639e-02 \\
ENGGVAL1 & 9.001882e-01 & 5.467747e-01 & 6.506329e-01 & 6.245341e-01 & 6.313226e-01 \\
LINEAR MINIMUM SURFACE & 8.027898e-01 & 1.836026e+00 & 3.193644e+00 & 4.812423e+00 & 7.008919e+00 \\
SQUARE ROOT 1 & 5.035099e-01 & -4.535902e-01 & -3.895292e-01 & 4.898633e-01 & -5.879180e-01 \\
SQUARE ROOT 2 & 1.638545e-01 & -7.905464e-01 & -5.497040e-01 & 3.584425e-01 & -7.701273e-01 \\
FREUDENTHAL ROTH & -8.150392e+00 & -4.169420e+17 & -4.092674e+17 & -4.092674e+17 & -4.092674e+17 \\
SPARSE MATRIX SQRT & 8.144777e-01 & -6.865589e-01 & 3.909375e-01 & -2.120778e-01 & -1.307948e-01 \\
ULTS0 & -1.769279e+10 & 2.817771e+10 & -2.424505e+10 & -8.518241e+09 & -5.897484e+09 \\
\hline
\hline
\end{tabular}
\end{table}
\end{landscape}

\section{Observações}

\begin{itemize}
\item O método L-BFGS-B foi configurado com tolerância de convergência de $10^{-6}$.
\item Para problemas que falharam, verifique a mensagem de erro específica.
\item A precisão é medida pela norma do gradiente ($||\nabla f(x^*)||$) calculada numericamente.
\item Valores de precisão menores indicam soluções mais próximas de pontos estacionários.
\item Para problemas irrestritos, $||\nabla f(x^*)|| \approx 0$ indica convergência para um mínimo local.
\item Problemas que falharam são marcados com "---" nas colunas de resultados.
\item A terceira tabela mostra as primeiras 5 variáveis da solução encontrada.
\item Para problemas com menos de 5 variáveis, as colunas extras são marcadas como "---".
\item A terceira tabela é apresentada em formato paisagem para melhor visualização.
\end{itemize}

\end{document}
