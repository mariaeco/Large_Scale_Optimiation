
\documentclass[12pt]{article}
\usepackage[utf8]{inputenc}
\usepackage[portuguese]{babel}
\usepackage{booktabs}
\usepackage{array}
\usepackage{geometry}
\usepackage{amsmath}
\usepackage{amsfonts}
\usepackage{pdflscape}

\geometry{a4paper, margin=2cm}

\title{Resultados dos Problemas de Otimização - Método Descida por Coordenadas (adaptive)}
\author{Análise Computacional}
\date{\today}

\begin{document}

\maketitle

\section{Problemas de Otimização}

A tabela 1 apresenta os problemas de otimização não-linear resolvidos usando o método Descida por Coordenadas (adaptive) e o número de variáveis de cada problema.

\begin{table}[h!]
\centering
\caption{Problemas de otimização e número de variáveis}
\label{tab:problemas_variáveis}
\begin{tabular}{@{}|lc|@{}}
\toprule
\textbf{Problema} & \textbf{Número de Variáveis} \\
\midrule
ROSENBROCK & 100 \\
PENALTY & 100 \\
TRIGONOMETRIC & 100 \\
EXTENDED ROSENBROCK & 100 \\
EXTENDED POWELL & 100 \\
QOR & 100 \\
GOR & 100 \\
PSP & 100 \\
TRIDIAGONAL & 100 \\
ENGGVAL1 & 100 \\
LINEAR MINIMUM SURFACE & 64 \\
SQUARE ROOT 1 & 36 \\
SQUARE ROOT 2 & 36 \\
FREUDENTHAL ROTH & 100 \\
SPARSE MATRIX SQRT & 16 \\
ULTS0 & 64 \\
\bottomrule
\end{tabular}
\end{table}

\section{Resultados de Convergência}

A tabela 2 apresenta os resultados de convergência para cada problema, incluindo o número de iterações necessárias, o valor mínimo da função objetivo encontrado e a precisão da solução (norma do gradiente).


\begin{table}[h!]
\small
\centering
\caption{Resultados de convergência dos problemas de otimização}
\label{tab:resultados_convergencia}
\small
\begin{tabular}{|l|cccc|}
\hline
\textbf{Problema} & \textbf{Iterações} & \textbf{Valor Mínimo} & \textbf{Precisão ($||\nabla f(x^*)||$)} & \textbf{Tempo (s)}\\
\hline
ROSENBROCK & 1 & 1.000e+00 & 2.000e+00 & 0.000s \\
PENALTY & 39 & 7.381e+00 & 4.232e-03 & 0.329s \\
TRIGONOMETRIC & 2 & 0.000e+00 & 2.000e-12 & 0.000s \\
EXTENDED ROSENBROCK & 293 & 6.059e+00 & 5.320e+00 & 25.420s \\
EXTENDED POWELL & 1 & 0.000e+00 & 0.000e+00 & 0.087s \\
QOR & 165 & 1.175e+03 & 5.635e-03 & 9.090s \\
GOR & 500 & 1.381e+03 & 7.230e-01 & 63.257s \\
PSP & 155 & 2.026e+02 & 2.066e+00 & 11.741s \\
TRIDIAGONAL & 177 & 1.495e-06 & 3.575e-03 & 2.197s \\
ENGGVAL1 & 72 & 1.091e+02 & 4.100e-03 & 5.161s \\
LINEAR MINIMUM SURFACE & 150 & 1.000e+00 & 3.540e-04 & 3.464s \\
SQUARE ROOT 1 & 398 & 5.701e-06 & 7.701e-04 & 8.345s \\
SQUARE ROOT 2 & 500 & 8.559e-06 & 1.936e-03 & 10.052s \\
FREUDENTHAL ROTH & 41 & 1.196e+04 & 5.929e-01 & 13.529s \\
SPARSE MATRIX SQRT & 15 & 1.865e-11 & 4.256e-06 & 0.217s \\
ULTS0 & 500 & 8.441e-02 & 2.784e+00 & 69.462s \\
\hline
\end{tabular}
\end{table}


\section{Soluções Encontradas (Primeiras 5 Variáveis)}

A tabela 3 apresenta as primeiras 5 variáveis da solução encontrada para cada problema. Para problemas com menos de 5 variáveis, apenas as variáveis disponíveis são mostradas.

\begin{landscape}
\begin{table}[h!]
\centering
\caption{Primeiras 5 variáveis das soluções encontradas}
\label{tab:solucoes_variáveis}
\begin{tabular}{|l|ccccc|}
\hline
\textbf{Problema} & \textbf{x1} & \textbf{x2} & \textbf{x3} & \textbf{x4} & \textbf{x5} \\
\hline
ROSENBROCK & 0.000000e+00 & 6.695529e-10 & 0.000000e+00 & 0.000000e+00 & 0.000000e+00 \\
PENALTY & 8.691405e-01 & 8.691392e-01 & 8.687964e-01 & 8.691367e-01 & 8.691411e-01 \\
TRIGONOMETRIC & 0.000000e+00 & --- & --- & --- & --- \\
EXTENDED ROSENBROCK & 9.999954e-01 & 9.999909e-01 & 9.999957e-01 & 9.999914e-01 & 9.999953e-01 \\
EXTENDED POWELL & 0.000000e+00 & 0.000000e+00 & 0.000000e+00 & 0.000000e+00 & 0.000000e+00 \\
QOR & 5.923215e-01 & -7.111567e-01 & 6.286632e-02 & -2.651200e+00 & 1.583493e+00 \\
GOR & -1.872929e+00 & -4.154783e-01 & -3.078713e+00 & -1.266546e-01 & 7.469774e+00 \\
PSP & 5.002248e+00 & 4.945659e+00 & 5.003007e+00 & 2.847757e+00 & 4.995841e+00 \\
TRIDIAGONAL & 1.000012e+00 & 5.000112e-01 & 2.500158e-01 & 1.250145e-01 & 6.251293e-02 \\
ENGGVAL1 & 9.010268e-01 & 5.458893e-01 & 6.511781e-01 & 6.241933e-01 & 6.315064e-01 \\
LINEAR MINIMUM SURFACE & 8.708513e-02 & 2.389030e-01 & 8.716331e-02 & 2.388221e-01 & 8.729167e-02 \\
SQUARE ROOT 1 & 6.236541e-01 & -2.392269e-01 & -2.159543e-01 & 4.868736e-01 & -6.005239e-01 \\
SQUARE ROOT 2 & 1.073077e+00 & -8.414184e-01 & -4.913646e-01 & 6.918138e-01 & -2.631390e-01 \\
FREUDENTHAL ROTH & 1.232439e+01 & -8.285013e-01 & -1.506708e+00 & -1.534662e+00 & -1.535798e+00 \\
SPARSE MATRIX SQRT & 8.414735e-01 & -7.568119e-01 & 4.121198e-01 & -2.879102e-01 & -1.323519e-01 \\
ULTS0 & 2.830303e-02 & -5.018067e-02 & -2.932794e-02 & -5.811265e-02 & 1.543456e-02 \\
\hline
\hline
\end{tabular}
\end{table}
\end{landscape}

\section{Observações}

\begin{itemize}
\item O método L-BFGS-B foi configurado com tolerância de convergência de $10^{-6}$.
\item Para problemas que falharam, verifique a mensagem de erro específica.
\item A precisão é medida pela norma do gradiente ($||\nabla f(x^*)||$) calculada numericamente.
\item Valores de precisão menores indicam soluções mais próximas de pontos estacionários.
\item Para problemas irrestritos, $||\nabla f(x^*)|| \approx 0$ indica convergência para um mínimo local.
\item Problemas que falharam são marcados com "---" nas colunas de resultados.
\item A terceira tabela mostra as primeiras 5 variáveis da solução encontrada.
\item Para problemas com menos de 5 variáveis, as colunas extras são marcadas como "---".
\item A terceira tabela é apresentada em formato paisagem para melhor visualização.
\end{itemize}

\end{document}
