
\documentclass[12pt]{article}
\usepackage[utf8]{inputenc}
\usepackage[portuguese]{babel}
\usepackage{booktabs}
\usepackage{array}
\usepackage{geometry}
\usepackage{amsmath}
\usepackage{amsfonts}
\usepackage{longtable}

\geometry{a4paper, margin=1.5cm}

\title{Solução do Problema Afiro - Solver HiGHS}
\author{Análise Computacional}
\date{\today}

\begin{document}

\maketitle

\section{Informações do Problema}

\textbf{Informações do Problema:}
\begin{itemize}
\item Nome: Afiro
\item Número de Variáveis: 32
\item Número de Restrições: 27
\item Inviabilidade Primal: 0.000e+00
\item Inviabilidade Dual: 0.000e+00
\item Valor Primal: -4.648e+02
\item Valor Dual: -4.648e+02
\item Gap: -1.314e-19
\item Número de Iterações: 7
\end{itemize}


\section{Variáveis Primais e Custos Reduzidos}

\begin{longtable}{@{}cccc@{}}
\caption{Variáveis primais e custos reduzidos do problema Afiro} \\
\toprule
\textbf{Coordenada x} & \textbf{Valor x} & \textbf{Coordenada z} & \textbf{Valor z} \\
\midrule
\endfirsthead

\toprule
\textbf{Coordenada x} & \textbf{Valor x} & \textbf{Coordenada z} & \textbf{Valor z} \\
\midrule
\endhead

\midrule \multicolumn{4}{r}{{Continua na próxima página}} \\ \midrule
\endfoot

\bottomrule
\endlastfoot
1 & 80.000000 & 1 & 0.000000 \\
2 & 25.500000 & 2 & 0.000000 \\
3 & 54.500000 & 3 & 0.000000 \\
4 & 84.800000 & 4 & 0.000000 \\
5 & 80.000000 & 5 & 0.000000 \\
6 & 0.000000 & 6 & 2.249657 \\
7 & 0.000000 & 7 & 2.270400 \\
8 & 0.000000 & 8 & 2.290200 \\
9 & 0.000000 & 9 & 2.228914 \\
10 & -0.000000 & 10 & 0.000000 \\
11 & -0.000000 & 11 & 0.000000 \\
12 & -0.000000 & 12 & 0.000000 \\
13 & 18.214286 & 13 & 0.000000 \\
14 & 61.785714 & 14 & 0.000000 \\
15 & 84.800000 & 15 & 0.000000 \\
16 & 500.000000 & 16 & 0.000000 \\
17 & 475.920000 & 17 & 0.000000 \\
18 & 24.080000 & 18 & 0.000000 \\
19 & -0.000000 & 19 & 0.000000 \\
20 & 215.000000 & 20 & 0.000000 \\
21 & 0.000000 & 21 & 0.000000 \\
22 & 0.000000 & 22 & 0.000000 \\
23 & 0.000000 & 23 & 0.000000 \\
24 & 0.000000 & 24 & 0.000000 \\
25 & 0.000000 & 25 & 2.065800 \\
26 & 0.000000 & 26 & 2.092200 \\
27 & 0.000000 & 27 & 2.120486 \\
28 & 0.000000 & 28 & 2.148771 \\
29 & 339.942857 & 29 & 0.000000 \\
30 & 383.942857 & 30 & 0.000000 \\
31 & -0.000000 & 31 & 0.000000 \\
32 & 0.000000 & 32 & 10.000000 \\

\end{longtable}

\section{Variáveis Duais (Multiplicadores de Lagrange)}

\begin{longtable}{@{}cc@{}}
\caption{Variáveis duais do problema Afiro} \\
\toprule
\textbf{Coordenada y} & \textbf{Valor y} \\
\midrule
\endfirsthead

\toprule
\textbf{Coordenada y} & \textbf{Valor y} \\
\midrule
\endhead

\midrule \multicolumn{2}{r}{{Continua na próxima página}} \\ \midrule
\endfoot

\bottomrule
\endlastfoot
1 & -0.628571 \\
2 & -0.000000 \\
3 & -0.344771 \\
4 & -0.228571 \\
5 & -0.000000 \\
6 & -0.000000 \\
7 & -0.000000 \\
8 & -2.249657 \\
9 & -2.270400 \\
10 & -2.290200 \\
11 & -0.942857 \\
12 & -0.000000 \\
13 & -0.874343 \\
14 & -0.342857 \\
15 & -0.000000 \\
16 & -0.000000 \\
17 & -0.000000 \\
18 & -0.000000 \\
19 & -0.000000 \\
20 & -0.000000 \\
21 & -0.942857 \\
22 & -0.628571 \\
23 & -0.000000 \\
24 & -0.942857 \\
25 & -0.000000 \\
26 & -0.000000 \\
27 & -0.000000 \\

\end{longtable}


\section{Observações}

\begin{itemize}
\item O solver HiGHS foi configurado com o método IPM (Interior Point Method).
\item Este arquivo contém a solução detalhada para o problema Afiro.
\item As variáveis duais representam os multiplicadores de Lagrange das restrições.
\item Os custos reduzidos (z) indicam o impacto de forçar variáveis não-básicas na base.
\end{itemize}

\end{document}
