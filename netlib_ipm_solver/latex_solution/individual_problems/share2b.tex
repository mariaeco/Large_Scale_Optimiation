
\documentclass[12pt]{article}
\usepackage[utf8]{inputenc}
\usepackage[portuguese]{babel}
\usepackage{booktabs}
\usepackage{array}
\usepackage{geometry}
\usepackage{amsmath}
\usepackage{amsfonts}
\usepackage{longtable}

\geometry{a4paper, margin=1.5cm}

\title{Solução do Problema share2b - Solver HiGHS}
\author{Análise Computacional}
\date{\today}

\begin{document}

\maketitle

\section{Informações do Problema}

\textbf{Informações do Problema:}
\begin{itemize}
\item Nome: share2b
\item Número de Variáveis: 79
\item Número de Restrições: 96
\item Inviabilidade Primal: 0.000e+00
\item Inviabilidade Dual: 0.000e+00
\item Valor Primal: -4.157e+02
\item Valor Dual: -4.157e+02
\item Gap: -9.855e-18
\item Número de Iterações: 15
\end{itemize}


\section{Variáveis Primais e Custos Reduzidos}

\begin{longtable}{@{}cccc@{}}
\caption{Variáveis primais e custos reduzidos do problema share2b} \\
\toprule
\textbf{Coordenada x} & \textbf{Valor x} & \textbf{Coordenada z} & \textbf{Valor z} \\
\midrule
\endfirsthead

\toprule
\textbf{Coordenada x} & \textbf{Valor x} & \textbf{Coordenada z} & \textbf{Valor z} \\
\midrule
\endhead

\midrule \multicolumn{4}{r}{{Continua na próxima página}} \\ \midrule
\endfoot

\bottomrule
\endlastfoot
1 & 1.958139 & 1 & 0.000000 \\
2 & 2.023227 & 2 & 0.000000 \\
3 & 0.000000 & 3 & 6.001917 \\
4 & 0.000000 & 4 & 5.463724 \\
5 & 0.000000 & 5 & 42.793431 \\
6 & 0.000000 & 6 & 20.236783 \\
7 & 5.833333 & 7 & 0.000000 \\
8 & 9.556750 & 8 & 0.000000 \\
9 & 58.114349 & 9 & 0.000000 \\
10 & 19.371450 & 10 & 0.000000 \\
11 & 4.371450 & 11 & 0.000000 \\
12 & 0.000000 & 12 & 3.700000 \\
13 & 1.728074 & 13 & 0.000000 \\
14 & 4.976773 & 14 & 0.000000 \\
15 & 4.561787 & 15 & 0.000000 \\
16 & 0.000000 & 16 & 27.402494 \\
17 & 0.000000 & 17 & 0.213011 \\
18 & 3.733366 & 18 & 0.000000 \\
19 & 14.187985 & 19 & 0.000000 \\
20 & 15.000000 & 20 & 0.000000 \\
21 & 1.176843 & 21 & 0.000000 \\
22 & 0.000000 & 22 & 16.054526 \\
23 & 0.000000 & 23 & 39.791979 \\
24 & 1.196347 & 24 & 0.000000 \\
25 & 0.000000 & 25 & 49.237145 \\
26 & 5.833333 & 26 & 0.000000 \\
27 & 0.000000 & 27 & 2.527913 \\
28 & 1.000000 & 28 & 0.000000 \\
29 & 0.000000 & 29 & 12.344498 \\
30 & 7.709884 & 30 & 0.000000 \\
31 & 50.749223 & 31 & 0.000000 \\
32 & 16.916408 & 32 & 0.000000 \\
33 & 6.287857 & 33 & 0.000000 \\
34 & -0.000000 & 34 & 0.000000 \\
35 & -0.000000 & 35 & 0.000000 \\
36 & -0.000000 & 36 & 0.000000 \\
37 & -0.000000 & 37 & 0.000000 \\
38 & 0.000000 & 38 & 9.630238 \\
39 & 0.000000 & 39 & 0.010544 \\
40 & 0.000000 & 40 & 9.252948 \\
41 & 0.000000 & 41 & 0.190647 \\
42 & -0.000000 & 42 & 0.000000 \\
43 & -0.000000 & 43 & 0.000000 \\
44 & 5.000000 & 44 & 0.000000 \\
45 & 1.005819 & 45 & 0.000000 \\
46 & 1.500000 & 46 & 0.000000 \\
47 & 0.456403 & 47 & 0.000000 \\
48 & 3.770878 & 48 & 0.000000 \\
49 & 0.000000 & 49 & 8.306801 \\
50 & 1.038253 & 50 & 0.000000 \\
51 & 0.000000 & 51 & 9.901836 \\
52 & 0.000000 & 52 & 0.326113 \\
53 & 5.940790 & 53 & 0.000000 \\
54 & 41.136428 & 54 & 0.000000 \\
55 & 13.712143 & 55 & 0.000000 \\
56 & 1.410030 & 56 & 0.000000 \\
57 & 6.229122 & 57 & 0.000000 \\
58 & 5.621123 & 58 & 0.000000 \\
59 & 0.000000 & 59 & 0.000000 \\
60 & 3.239725 & 60 & 0.000000 \\
61 & 1.500000 & 61 & 0.000000 \\
62 & 1.000000 & 62 & 0.000000 \\
63 & 1.000000 & 63 & 0.000000 \\
64 & 0.000000 & 64 & 0.000000 \\
65 & 19.703821 & 65 & 0.000000 \\
66 & 20.000000 & 66 & 0.000000 \\
67 & 0.000000 & 67 & 2.800000 \\
68 & 20.000000 & 68 & 0.000000 \\
69 & 1.457935 & 69 & 0.000000 \\
70 & 0.000000 & 70 & 0.000000 \\
71 & 4.378877 & 71 & 0.000000 \\
72 & 1.500000 & 72 & 0.000000 \\
73 & 4.222022 & 73 & 0.000000 \\
74 & 0.000000 & 74 & 0.000000 \\
75 & 0.000000 & 75 & 0.000000 \\
76 & 5.337320 & 76 & 0.000000 \\
77 & 0.000000 & 77 & 0.250438 \\
78 & 6.957063 & 78 & 0.000000 \\
79 & 16.896153 & 79 & 0.000000 \\

\end{longtable}

\section{Variáveis Duais (Multiplicadores de Lagrange)}

\begin{longtable}{@{}cc@{}}
\caption{Variáveis duais do problema share2b} \\
\toprule
\textbf{Coordenada y} & \textbf{Valor y} \\
\midrule
\endfirsthead

\toprule
\textbf{Coordenada y} & \textbf{Valor y} \\
\midrule
\endhead

\midrule \multicolumn{2}{r}{{Continua na próxima página}} \\ \midrule
\endfoot

\bottomrule
\endlastfoot
1 & -0.125643 \\
2 & -0.000000 \\
3 & -0.000000 \\
4 & -0.000000 \\
5 & -0.000000 \\
6 & -0.332508 \\
7 & -0.000000 \\
8 & -1.912622 \\
9 & -2.055378 \\
10 & -187.623716 \\
11 & -0.003752 \\
12 & -0.000000 \\
13 & -0.000000 \\
14 & -0.000000 \\
15 & -0.000000 \\
16 & -0.000000 \\
17 & -0.026471 \\
18 & -0.000000 \\
19 & -0.000000 \\
20 & -1.971704 \\
21 & -0.686000 \\
22 & -0.000000 \\
23 & -0.000000 \\
24 & -0.000000 \\
25 & -0.000000 \\
26 & -1.427656 \\
27 & -0.000000 \\
28 & -2.175262 \\
29 & -3.152626 \\
30 & -315.120359 \\
31 & -0.004968 \\
32 & -0.309311 \\
33 & -0.000000 \\
34 & -0.000000 \\
35 & -0.007709 \\
36 & -0.000000 \\
37 & -0.019109 \\
38 & -0.000000 \\
39 & -0.006945 \\
40 & -1.908343 \\
41 & -0.340607 \\
42 & -0.000000 \\
43 & -10.498548 \\
44 & -0.000000 \\
45 & -0.000000 \\
46 & -0.192137 \\
47 & -0.000000 \\
48 & -0.565739 \\
49 & -0.516301 \\
50 & -68.467458 \\
51 & -0.064053 \\
52 & -0.000000 \\
53 & -0.000000 \\
54 & -0.000000 \\
55 & -0.000000 \\
56 & -0.000000 \\
57 & -0.021429 \\
58 & -0.000000 \\
59 & -0.000000 \\
60 & 1.561957 \\
61 & -0.109424 \\
62 & -0.000000 \\
63 & -27.241044 \\
64 & -0.145357 \\
65 & -0.000000 \\
66 & -0.000000 \\
67 & -0.000000 \\
68 & -9.078053 \\
69 & -0.567194 \\
70 & -0.000000 \\
71 & -2.567283 \\
72 & -3.053967 \\
73 & -2.496778 \\
74 & -2.811759 \\
75 & -2.940652 \\
76 & -2.882422 \\
77 & -2.283214 \\
78 & -2.662663 \\
79 & 6.414937 \\
80 & 2.714937 \\
81 & -0.785063 \\
82 & -3.528162 \\
83 & -2.785481 \\
84 & -0.085481 \\
85 & -38.062205 \\
86 & -2.743093 \\
87 & -0.064053 \\
88 & -0.000000 \\
89 & -0.000000 \\
90 & -0.000000 \\
91 & -0.000000 \\
92 & -0.000000 \\
93 & -0.021429 \\
94 & -0.000000 \\
95 & -0.000000 \\
96 & 1.561957 \\

\end{longtable}


\section{Observações}

\begin{itemize}
\item O solver HiGHS foi configurado com o método IPM (Interior Point Method).
\item Este arquivo contém a solução detalhada para o problema share2b.
\item As variáveis duais representam os multiplicadores de Lagrange das restrições.
\item Os custos reduzidos (z) indicam o impacto de forçar variáveis não-básicas na base.
\end{itemize}

\end{document}
