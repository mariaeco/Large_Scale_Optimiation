
\documentclass[12pt]{article}
\usepackage[utf8]{inputenc}
\usepackage[portuguese]{babel}
\usepackage{booktabs}
\usepackage{array}
\usepackage{geometry}
\usepackage{amsmath}
\usepackage{amsfonts}
\usepackage{longtable}

\geometry{a4paper, margin=1.5cm}

\title{Solução do Problema Blend - Solver HiGHS}
\author{Análise Computacional}
\date{\today}

\begin{document}

\maketitle

\section{Informações do Problema}

\textbf{Informações do Problema:}
\begin{itemize}
\item Nome: Blend
\item Número de Variáveis: 83
\item Número de Restrições: 74
\item Inviabilidade Primal: 0.000e+00
\item Inviabilidade Dual: 4.025e-15
\item Valor Primal: -3.081e+01
\item Valor Dual: -3.081e+01
\item Gap: -3.682e-18
\item Número de Iterações: 10
\end{itemize}


\section{Variáveis Primais e Custos Reduzidos}

\begin{longtable}{@{}cccc@{}}
\caption{Variáveis primais e custos reduzidos do problema Blend} \\
\toprule
\textbf{Coordenada x} & \textbf{Valor x} & \textbf{Coordenada z} & \textbf{Valor z} \\
\midrule
\endfirsthead

\toprule
\textbf{Coordenada x} & \textbf{Valor x} & \textbf{Coordenada z} & \textbf{Valor z} \\
\midrule
\endhead

\midrule \multicolumn{4}{r}{{Continua na próxima página}} \\ \midrule
\endfoot

\bottomrule
\endlastfoot
1 & 20.944802 & 1 & 0.000000 \\
2 & 10.170922 & 2 & 0.000000 \\
3 & 11.247359 & 3 & 0.000000 \\
4 & 2.981097 & 4 & 0.000000 \\
5 & 0.659704 & 5 & 0.000000 \\
6 & 0.475926 & 6 & 0.000000 \\
7 & -0.000000 & 7 & 0.000000 \\
8 & 10.101176 & 8 & 0.000000 \\
9 & 0.000000 & 9 & 0.000000 \\
10 & 1.679179 & 10 & 0.000000 \\
11 & -0.000000 & 11 & 0.000000 \\
12 & 10.101176 & 12 & 0.000000 \\
13 & -0.000000 & 13 & 0.000000 \\
14 & 11.780355 & 14 & 0.000000 \\
15 & 0.000000 & 15 & -0.000000 \\
16 & 0.406743 & 16 & 0.000000 \\
17 & 0.000000 & 17 & 0.155780 \\
18 & 2.173257 & 18 & 0.000000 \\
19 & 2.018560 & 19 & 0.000000 \\
20 & 4.843257 & 20 & 0.000000 \\
21 & 0.000000 & 21 & -0.000000 \\
22 & 3.138183 & 22 & 0.000000 \\
23 & 1.149910 & 23 & 0.000000 \\
24 & 1.396291 & 24 & 0.000000 \\
25 & 0.000000 & 25 & 1.985268 \\
26 & 0.000000 & 26 & 3.113157 \\
27 & 0.000000 & 27 & 0.500000 \\
28 & 0.000000 & 28 & 0.088756 \\
29 & 0.383500 & 29 & 0.000000 \\
30 & 0.000000 & 30 & 0.092427 \\
31 & 4.424431 & 31 & 0.000000 \\
32 & 0.000000 & 32 & 0.099720 \\
33 & 1.149910 & 33 & 0.000000 \\
34 & 1.396291 & 34 & 0.000000 \\
35 & 0.748570 & 35 & 0.000000 \\
36 & 21.638387 & 36 & 0.000000 \\
37 & 8.102703 & 37 & 0.000000 \\
38 & 0.710776 & 38 & 0.000000 \\
39 & 0.481789 & 39 & 0.000000 \\
40 & 0.000000 & 40 & 0.121113 \\
41 & 4.892936 & 41 & 0.000000 \\
42 & 0.000000 & 42 & 0.070597 \\
43 & 0.441675 & 43 & 0.000000 \\
44 & 14.285737 & 44 & 0.000000 \\
45 & 6.527177 & 45 & 0.000000 \\
46 & 2.005817 & 46 & 0.000000 \\
47 & 0.000000 & 47 & 0.145701 \\
48 & 0.000000 & 48 & 0.147905 \\
49 & 0.771329 & 49 & 0.000000 \\
50 & 4.876257 & 50 & 0.000000 \\
51 & 0.224225 & 51 & 0.000000 \\
52 & 1.811156 & 52 & 0.000000 \\
53 & 7.877628 & 53 & 0.000000 \\
54 & 0.320155 & 54 & 0.000000 \\
55 & 0.989247 & 55 & 0.000000 \\
56 & 0.443908 & 56 & 0.000000 \\
57 & 1.433155 & 57 & 0.000000 \\
58 & -0.000000 & 58 & 0.000000 \\
59 & -0.000000 & 59 & 0.000000 \\
60 & -0.000000 & 60 & 0.000000 \\
61 & 3.079217 & 61 & 0.000000 \\
62 & 0.795584 & 62 & 0.000000 \\
63 & 3.874801 & 63 & 0.000000 \\
64 & 0.774958 & 64 & 0.000000 \\
65 & 1.830766 & 65 & 0.000000 \\
66 & 0.000000 & 66 & 1.177242 \\
67 & 0.000000 & 67 & 0.885864 \\
68 & 0.065873 & 68 & 0.000000 \\
69 & 0.788912 & 69 & 0.000000 \\
70 & 3.460509 & 70 & 0.000000 \\
71 & 2.750889 & 71 & 0.000000 \\
72 & 0.000000 & 72 & 3.250000 \\
73 & 0.000000 & 73 & 2.750000 \\
74 & 0.000000 & 74 & 2.750000 \\
75 & 0.000000 & 75 & 4.735268 \\
76 & 0.000000 & 76 & 5.863157 \\
77 & 0.169396 & 77 & 0.000000 \\
78 & 0.000000 & 78 & 0.040000 \\
79 & 1.154801 & 79 & 0.000000 \\
80 & 0.000000 & 80 & 0.400000 \\
81 & 0.803301 & 81 & 0.000000 \\
82 & 26.030369 & 82 & 0.000000 \\
83 & 87.094974 & 83 & 0.000000 \\

\end{longtable}

\section{Variáveis Duais (Multiplicadores de Lagrange)}

\begin{longtable}{@{}cc@{}}
\caption{Variáveis duais do problema Blend} \\
\toprule
\textbf{Coordenada y} & \textbf{Valor y} \\
\midrule
\endfirsthead

\toprule
\textbf{Coordenada y} & \textbf{Valor y} \\
\midrule
\endhead

\midrule \multicolumn{2}{r}{{Continua na próxima página}} \\ \midrule
\endfoot

\bottomrule
\endlastfoot
1 & -3.632654 \\
2 & -3.492029 \\
3 & -2.938966 \\
4 & -2.938966 \\
5 & -3.035620 \\
6 & -2.941574 \\
7 & -2.000000 \\
8 & -1.358969 \\
9 & -2.750000 \\
10 & -2.750000 \\
11 & -4.333928 \\
12 & -4.383227 \\
13 & -3.250000 \\
14 & -4.396151 \\
15 & -4.396151 \\
16 & -4.396151 \\
17 & -2.592643 \\
18 & -2.614110 \\
19 & -5.262400 \\
20 & -5.029973 \\
21 & -2.815620 \\
22 & -4.396151 \\
23 & -4.748524 \\
24 & -5.863157 \\
25 & -4.735268 \\
26 & -2.750000 \\
27 & -4.969502 \\
28 & -3.444220 \\
29 & -2.519412 \\
30 & -5.368449 \\
31 & -5.475108 \\
32 & 0.043360 \\
33 & -1.082121 \\
34 & -2.938966 \\
35 & -4.559056 \\
36 & -3.211811 \\
37 & -2.750000 \\
38 & -4.386215 \\
39 & -0.000000 \\
40 & -0.040000 \\
41 & -0.000000 \\
42 & -0.013200 \\
43 & -0.010000 \\
44 & -0.000000 \\
45 & -0.560000 \\
46 & -0.000000 \\
47 & -0.041676 \\
48 & -0.000000 \\
49 & -0.000000 \\
50 & -0.000000 \\
51 & -0.444231 \\
52 & -0.000000 \\
53 & -0.040712 \\
54 & -0.000000 \\
55 & -0.010269 \\
56 & -0.216901 \\
57 & -0.000000 \\
58 & -0.000000 \\
59 & -0.037927 \\
60 & -0.079954 \\
61 & -0.000000 \\
62 & -0.253372 \\
63 & -0.175011 \\
64 & -0.178746 \\
65 & -0.661034 \\
66 & -0.755080 \\
67 & -0.000000 \\
68 & -0.000000 \\
69 & -0.000000 \\
70 & -1.228115 \\
71 & -0.830378 \\
72 & -0.000000 \\
73 & -0.089807 \\
74 & -0.357167 \\

\end{longtable}


\section{Observações}

\begin{itemize}
\item O solver HiGHS foi configurado com o método IPM (Interior Point Method).
\item Este arquivo contém a solução detalhada para o problema Blend.
\item As variáveis duais representam os multiplicadores de Lagrange das restrições.
\item Os custos reduzidos (z) indicam o impacto de forçar variáveis não-básicas na base.
\end{itemize}

\end{document}
