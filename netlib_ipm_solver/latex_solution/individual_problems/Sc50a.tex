
\documentclass[12pt]{article}
\usepackage[utf8]{inputenc}
\usepackage[portuguese]{babel}
\usepackage{booktabs}
\usepackage{array}
\usepackage{geometry}
\usepackage{amsmath}
\usepackage{amsfonts}
\usepackage{longtable}

\geometry{a4paper, margin=1.5cm}

\title{Solução do Problema Sc50a - Solver HiGHS}
\author{Análise Computacional}
\date{\today}

\begin{document}

\maketitle

\section{Informações do Problema}

\textbf{Informações do Problema:}
\begin{itemize}
\item Nome: Sc50a
\item Número de Variáveis: 48
\item Número de Restrições: 50
\item Inviabilidade Primal: 0.000e+00
\item Inviabilidade Dual: 0.000e+00
\item Valor Primal: -6.458e+01
\item Valor Dual: -6.458e+01
\item Gap: -0.000e+00
\item Número de Iterações: 8
\end{itemize}


\section{Variáveis Primais e Custos Reduzidos}

\begin{longtable}{@{}cccc@{}}
\caption{Variáveis primais e custos reduzidos do problema Sc50a} \\
\toprule
\textbf{Coordenada x} & \textbf{Valor x} & \textbf{Coordenada z} & \textbf{Valor z} \\
\midrule
\endfirsthead

\toprule
\textbf{Coordenada x} & \textbf{Valor x} & \textbf{Coordenada z} & \textbf{Valor z} \\
\midrule
\endhead

\midrule \multicolumn{4}{r}{{Continua na próxima página}} \\ \midrule
\endfoot

\bottomrule
\endlastfoot
1 & -0.000000 & 1 & 0.000000 \\
2 & 16.568692 & 2 & 0.000000 \\
3 & 64.575077 & 3 & 0.000000 \\
4 & 64.575077 & 4 & 0.000000 \\
5 & 64.575077 & 5 & 0.000000 \\
6 & -0.000000 & 6 & 0.000000 \\
7 & 16.568692 & 7 & 0.000000 \\
8 & 64.575077 & 8 & 0.000000 \\
9 & 0.000000 & 9 & 0.077059 \\
10 & 16.568692 & 10 & 0.000000 \\
11 & 64.575077 & 11 & 0.000000 \\
12 & 0.000000 & 12 & 0.030823 \\
13 & 20.009908 & 13 & 0.000000 \\
14 & 71.032585 & 14 & 0.000000 \\
15 & 71.032585 & 15 & 0.000000 \\
16 & 135.607662 & 16 & 0.000000 \\
17 & -0.000000 & 17 & 0.000000 \\
18 & 36.578600 & 18 & 0.000000 \\
19 & 135.607662 & 19 & 0.000000 \\
20 & -0.000000 & 20 & 0.000000 \\
21 & 36.578600 & 21 & 0.000000 \\
22 & 135.607662 & 22 & 0.000000 \\
23 & 14.177345 & 23 & 0.000000 \\
24 & 17.598745 & 24 & 0.000000 \\
25 & 78.135843 & 25 & 0.000000 \\
26 & 78.135843 & 26 & 0.000000 \\
27 & 213.743505 & 27 & 0.000000 \\
28 & 14.177345 & 28 & 0.000000 \\
29 & 54.177345 & 29 & 0.000000 \\
30 & 213.743505 & 30 & 0.000000 \\
31 & 14.177345 & 31 & 0.000000 \\
32 & 54.177345 & 32 & 0.000000 \\
33 & 213.743505 & 33 & 0.000000 \\
34 & 18.417734 & 34 & 0.000000 \\
35 & 18.417734 & 35 & 0.000000 \\
36 & 85.949428 & 36 & 0.000000 \\
37 & 85.949428 & 37 & 0.000000 \\
38 & 299.692933 & 38 & 0.000000 \\
39 & 32.595079 & 39 & 0.000000 \\
40 & 72.595079 & 40 & 0.000000 \\
41 & 299.692933 & 41 & 0.000000 \\
42 & 32.595079 & 42 & 0.000000 \\
43 & 72.595079 & 43 & 0.000000 \\
44 & 299.692933 & 44 & 0.000000 \\
45 & 20.259508 & 45 & 0.000000 \\
46 & 20.259508 & 46 & 0.000000 \\
47 & 94.544370 & 47 & 0.000000 \\
48 & 94.544370 & 48 & 0.000000 \\

\end{longtable}

\section{Variáveis Duais (Multiplicadores de Lagrange)}

\begin{longtable}{@{}cc@{}}
\caption{Variáveis duais do problema Sc50a} \\
\toprule
\textbf{Coordenada y} & \textbf{Valor y} \\
\midrule
\endfirsthead

\toprule
\textbf{Coordenada y} & \textbf{Valor y} \\
\midrule
\endhead

\midrule \multicolumn{2}{r}{{Continua na próxima página}} \\ \midrule
\endfoot

\bottomrule
\endlastfoot
1 & -0.000000 \\
2 & -0.138705 \\
3 & -0.000000 \\
4 & -0.208058 \\
5 & -0.138705 \\
6 & -0.277411 \\
7 & -0.208058 \\
8 & -0.077059 \\
9 & -0.092470 \\
10 & -0.069353 \\
11 & -0.000000 \\
12 & -0.092470 \\
13 & -0.069353 \\
14 & -0.719947 \\
15 & -0.138705 \\
16 & -0.061647 \\
17 & -0.184941 \\
18 & -0.138705 \\
19 & -0.000000 \\
20 & -0.061647 \\
21 & -0.046235 \\
22 & -0.000000 \\
23 & -0.061647 \\
24 & -0.046235 \\
25 & -0.528402 \\
26 & -0.092470 \\
27 & -0.061647 \\
28 & -0.123294 \\
29 & -0.092470 \\
30 & -0.007706 \\
31 & -0.038529 \\
32 & -0.023118 \\
33 & -0.007706 \\
34 & -0.038529 \\
35 & -0.023118 \\
36 & -0.396301 \\
37 & -0.069353 \\
38 & -0.053941 \\
39 & -0.084764 \\
40 & -0.069353 \\
41 & -0.053941 \\
42 & -0.084764 \\
43 & -0.069353 \\
44 & -0.053941 \\
45 & -0.084764 \\
46 & -0.069353 \\
47 & -0.297226 \\
48 & -0.280102 \\
49 & -0.314350 \\
50 & -0.297226 \\

\end{longtable}


\section{Observações}

\begin{itemize}
\item O solver HiGHS foi configurado com o método IPM (Interior Point Method).
\item Este arquivo contém a solução detalhada para o problema Sc50a.
\item As variáveis duais representam os multiplicadores de Lagrange das restrições.
\item Os custos reduzidos (z) indicam o impacto de forçar variáveis não-básicas na base.
\end{itemize}

\end{document}
