
\documentclass[12pt]{article}
\usepackage[utf8]{inputenc}
\usepackage[portuguese]{babel}
\usepackage{booktabs}
\usepackage{array}
\usepackage{geometry}
\usepackage{amsmath}
\usepackage{amsfonts}
\usepackage{longtable}

\geometry{a4paper, margin=1.5cm}

\title{Solução do Problema Sc50b - Solver HiGHS}
\author{Análise Computacional}
\date{\today}

\begin{document}

\maketitle

\section{Informações do Problema}

\textbf{Informações do Problema:}
\begin{itemize}
\item Nome: Sc50b
\item Número de Variáveis: 48
\item Número de Restrições: 50
\item Inviabilidade Primal: 0.000e+00
\item Inviabilidade Dual: 0.000e+00
\item Valor Primal: -7.000e+01
\item Valor Dual: -7.000e+01
\item Gap: -1.440e-18
\item Número de Iterações: 8
\end{itemize}


\section{Variáveis Primais e Custos Reduzidos}

\begin{longtable}{@{}cccc@{}}
\caption{Variáveis primais e custos reduzidos do problema Sc50b} \\
\toprule
\textbf{Coordenada x} & \textbf{Valor x} & \textbf{Coordenada z} & \textbf{Valor z} \\
\midrule
\endfirsthead

\toprule
\textbf{Coordenada x} & \textbf{Valor x} & \textbf{Coordenada z} & \textbf{Valor z} \\
\midrule
\endhead

\midrule \multicolumn{4}{r}{{Continua na próxima página}} \\ \midrule
\endfoot

\bottomrule
\endlastfoot
1 & 30.000000 & 1 & 0.000000 \\
2 & 28.000000 & 2 & 0.000000 \\
3 & 42.000000 & 3 & 0.000000 \\
4 & 70.000000 & 4 & 0.000000 \\
5 & 70.000000 & 5 & 0.000000 \\
6 & 30.000000 & 6 & 0.000000 \\
7 & 28.000000 & 7 & 0.000000 \\
8 & 42.000000 & 8 & 0.000000 \\
9 & 30.000000 & 9 & 0.000000 \\
10 & 28.000000 & 10 & 0.000000 \\
11 & 42.000000 & 11 & 0.000000 \\
12 & 33.000000 & 12 & 0.000000 \\
13 & 30.800000 & 13 & 0.000000 \\
14 & 46.200000 & 14 & 0.000000 \\
15 & 77.000000 & 15 & 0.000000 \\
16 & 147.000000 & 16 & 0.000000 \\
17 & 63.000000 & 17 & 0.000000 \\
18 & 58.800000 & 18 & 0.000000 \\
19 & 88.200000 & 19 & 0.000000 \\
20 & 63.000000 & 20 & 0.000000 \\
21 & 58.800000 & 21 & 0.000000 \\
22 & 88.200000 & 22 & 0.000000 \\
23 & 36.300000 & 23 & 0.000000 \\
24 & 33.880000 & 24 & 0.000000 \\
25 & 50.820000 & 25 & 0.000000 \\
26 & 84.700000 & 26 & 0.000000 \\
27 & 231.700000 & 27 & 0.000000 \\
28 & 99.300000 & 28 & 0.000000 \\
29 & 92.680000 & 29 & 0.000000 \\
30 & 139.020000 & 30 & 0.000000 \\
31 & 99.300000 & 31 & 0.000000 \\
32 & 92.680000 & 32 & 0.000000 \\
33 & 139.020000 & 33 & 0.000000 \\
34 & 39.930000 & 34 & 0.000000 \\
35 & 37.268000 & 35 & 0.000000 \\
36 & 55.902000 & 36 & 0.000000 \\
37 & 93.170000 & 37 & 0.000000 \\
38 & 324.870000 & 38 & 0.000000 \\
39 & 139.230000 & 39 & 0.000000 \\
40 & 129.948000 & 40 & 0.000000 \\
41 & 194.922000 & 41 & 0.000000 \\
42 & 139.230000 & 42 & 0.000000 \\
43 & 129.948000 & 43 & 0.000000 \\
44 & 194.922000 & 44 & 0.000000 \\
45 & 43.923000 & 45 & 0.000000 \\
46 & 40.994800 & 46 & 0.000000 \\
47 & 61.492200 & 47 & 0.000000 \\
48 & 102.487000 & 48 & 0.000000 \\

\end{longtable}

\section{Variáveis Duais (Multiplicadores de Lagrange)}

\begin{longtable}{@{}cc@{}}
\caption{Variáveis duais do problema Sc50b} \\
\toprule
\textbf{Coordenada y} & \textbf{Valor y} \\
\midrule
\endfirsthead

\toprule
\textbf{Coordenada y} & \textbf{Valor y} \\
\midrule
\endhead

\midrule \multicolumn{2}{r}{{Continua na próxima página}} \\ \midrule
\endfoot

\bottomrule
\endlastfoot
1 & -0.058333 \\
2 & -0.000000 \\
3 & -0.000000 \\
4 & -0.175000 \\
5 & -0.175000 \\
6 & -0.175000 \\
7 & -0.175000 \\
8 & -0.043750 \\
9 & -0.043750 \\
10 & -0.043750 \\
11 & -0.043750 \\
12 & -0.043750 \\
13 & -0.043750 \\
14 & -0.750000 \\
15 & -0.131250 \\
16 & -0.131250 \\
17 & -0.131250 \\
18 & -0.131250 \\
19 & -0.032812 \\
20 & -0.032812 \\
21 & -0.032812 \\
22 & -0.032812 \\
23 & -0.032812 \\
24 & -0.032812 \\
25 & -0.562500 \\
26 & -0.098437 \\
27 & -0.098437 \\
28 & -0.098437 \\
29 & -0.098437 \\
30 & -0.024609 \\
31 & -0.024609 \\
32 & -0.024609 \\
33 & -0.024609 \\
34 & -0.024609 \\
35 & -0.024609 \\
36 & -0.421875 \\
37 & -0.073828 \\
38 & -0.073828 \\
39 & -0.073828 \\
40 & -0.073828 \\
41 & -0.073828 \\
42 & -0.073828 \\
43 & -0.073828 \\
44 & -0.073828 \\
45 & -0.073828 \\
46 & -0.073828 \\
47 & -0.316406 \\
48 & -0.316406 \\
49 & -0.316406 \\
50 & -0.316406 \\

\end{longtable}


\section{Observações}

\begin{itemize}
\item O solver HiGHS foi configurado com o método IPM (Interior Point Method).
\item Este arquivo contém a solução detalhada para o problema Sc50b.
\item As variáveis duais representam os multiplicadores de Lagrange das restrições.
\item Os custos reduzidos (z) indicam o impacto de forçar variáveis não-básicas na base.
\end{itemize}

\end{document}
