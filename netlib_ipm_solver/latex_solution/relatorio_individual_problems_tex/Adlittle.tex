
\documentclass[12pt]{article}
\usepackage[utf8]{inputenc}
\usepackage[portuguese]{babel}
\usepackage{booktabs}
\usepackage{array}
\usepackage{geometry}
\usepackage{amsmath}
\usepackage{amsfonts}
\usepackage{longtable}

\geometry{a4paper, margin=1.5cm}

\title{Solução do Problema Adlittle - Solver HiGHS}
\author{Análise Computacional}
\date{\today}

\begin{document}

\maketitle

\section{Informações do Problema}

\textbf{Informações do Problema:}
\begin{itemize}
\item Nome: Adlittle
\item Número de Variáveis: 97
\item Número de Restrições: 56
\item Inviabilidade Primal: 0.000e+00
\item Inviabilidade Dual: 4.547e-13
\item Valor Primal: 2.255e+05
\item Valor Dual: 2.255e+05
\item Gap: 2.862e-22
\item Número de Iterações: 13
\end{itemize}


\section{Variáveis Primais e Custos Reduzidos}

\begin{longtable}{@{}cccc@{}}
\caption{Variáveis primais e custos reduzidos do problema Adlittle} \\
\toprule
\textbf{Coordenada x} & \textbf{Valor x} & \textbf{Coordenada z} & \textbf{Valor z} \\
\midrule
\endfirsthead

\toprule
\textbf{Coordenada x} & \textbf{Valor x} & \textbf{Coordenada z} & \textbf{Valor z} \\
\midrule
\endhead

\midrule \multicolumn{4}{r}{{Continua na próxima página}} \\ \midrule
\endfoot

\bottomrule
\endlastfoot
1 & 22.854545 & 1 & 0.000000 \\
2 & 0.545455 & 2 & 0.000000 \\
3 & 4.626937 & 3 & 0.000000 \\
4 & 0.000000 & 4 & 2340.955570 \\
5 & 0.000000 & 5 & 10.973786 \\
6 & 35.107143 & 6 & 0.000000 \\
7 & 0.000000 & 7 & 0.000000 \\
8 & 7.735816 & 8 & 0.000000 \\
9 & 0.000000 & 9 & 478.196421 \\
10 & 44.405550 & 10 & 0.000000 \\
11 & 0.000000 & 11 & -0.000000 \\
12 & 9.884315 & 12 & 0.000000 \\
13 & 0.000000 & 13 & -0.000000 \\
14 & 0.000000 & 14 & -0.000000 \\
15 & 108.000000 & 15 & 0.000000 \\
16 & 0.000000 & 16 & 0.391753 \\
17 & 0.000000 & 17 & 0.391753 \\
18 & 0.000000 & 18 & 0.391753 \\
19 & 0.000000 & 19 & -0.000000 \\
20 & 0.000000 & 20 & -0.000000 \\
21 & 13.000000 & 21 & 0.000000 \\
22 & 0.000000 & 22 & 0.061856 \\
23 & 0.000000 & 23 & 0.061856 \\
24 & 0.000000 & 24 & 0.061856 \\
25 & 73.928496 & 25 & 0.000000 \\
26 & 51.294986 & 26 & 0.000000 \\
27 & 139.776519 & 27 & 0.000000 \\
28 & 0.000000 & 28 & -0.000000 \\
29 & 0.000000 & 29 & -0.000000 \\
30 & 109.339166 & 30 & 0.000000 \\
31 & 5.946942 & 31 & 0.000000 \\
32 & 1.776908 & 32 & 0.000000 \\
33 & 134.000000 & 33 & 0.000000 \\
34 & 0.000000 & 34 & 478.196421 \\
35 & 0.000000 & 35 & 0.000000 \\
36 & 0.000000 & 36 & -0.000000 \\
37 & 31.000000 & 37 & 0.000000 \\
38 & 0.000000 & 38 & 0.000000 \\
39 & 0.000000 & 39 & -0.000000 \\
40 & 0.000000 & 40 & -0.000000 \\
41 & 0.000000 & 41 & -0.000000 \\
42 & 60.000000 & 42 & 0.000000 \\
43 & 111.727273 & 43 & 0.000000 \\
44 & 0.000000 & 44 & 15.992309 \\
45 & 51.909091 & 45 & 0.000000 \\
46 & 29.863308 & 46 & 0.000000 \\
47 & 4.136692 & 47 & 0.000000 \\
48 & 0.000000 & 48 & -0.000000 \\
49 & 0.000000 & 49 & 176.126012 \\
50 & 41.500000 & 50 & 0.000000 \\
51 & 0.000000 & 51 & 3.136142 \\
52 & 15.478817 & 52 & 0.000000 \\
53 & 0.000000 & 53 & 51.578926 \\
54 & 15.000000 & 54 & 0.000000 \\
55 & 0.000000 & 55 & 13.398768 \\
56 & 0.000000 & 56 & 508.903385 \\
57 & 3.100000 & 57 & 0.000000 \\
58 & 0.000000 & 58 & 4.807255 \\
59 & 0.690909 & 59 & 0.000000 \\
60 & 0.000000 & 60 & 18.413913 \\
61 & 0.000000 & 61 & 0.000000 \\
62 & 175.444717 & 62 & 0.000000 \\
63 & 0.000000 & 63 & 11.704395 \\
64 & 9.806141 & 64 & 0.000000 \\
65 & 0.000000 & 65 & 162.614756 \\
66 & 0.000000 & 66 & 640.811177 \\
67 & 33.468370 & 67 & 0.000000 \\
68 & 9.531630 & 68 & 0.000000 \\
69 & 10.293075 & 69 & 0.000000 \\
70 & 8.906925 & 70 & 0.000000 \\
71 & 0.000000 & 71 & 0.377848 \\
72 & 0.000000 & 72 & 19.115087 \\
73 & 6.100000 & 73 & 0.000000 \\
74 & 0.000000 & 74 & 13.124360 \\
75 & 9.792857 & 75 & 0.000000 \\
76 & 313.197353 & 76 & 0.000000 \\
77 & 55.431678 & 77 & 0.000000 \\
78 & 264.555283 & 78 & 0.000000 \\
79 & 0.000000 & 79 & 289.787200 \\
80 & 13.200000 & 80 & 0.000000 \\
81 & 0.000000 & 81 & 3.832246 \\
82 & 1.314480 & 82 & 0.000000 \\
83 & 0.503261 & 83 & 0.000000 \\
84 & 0.000000 & 84 & 1028.417096 \\
85 & 0.000000 & 85 & 920.048061 \\
86 & 0.000000 & 86 & 460.634556 \\
87 & 0.000000 & 87 & 1011.855458 \\
88 & 0.000000 & 88 & 1287.845736 \\
89 & 0.000000 & 89 & 1000.000000 \\
90 & 0.000000 & 90 & 1000.000000 \\
91 & 2.649797 & 91 & 0.000000 \\
92 & 13.500000 & 92 & 0.000000 \\
93 & 6.087276 & 93 & 0.000000 \\
94 & 0.000000 & 94 & 844.183076 \\
95 & 31.200000 & 95 & 0.000000 \\
96 & -0.000000 & 96 & 0.000000 \\
97 & 0.000000 & 97 & 478.196421 \\

\end{longtable}

\section{Variáveis Duais (Multiplicadores de Lagrange)}

\begin{longtable}{@{}cc@{}}
\caption{Variáveis duais do problema Adlittle} \\
\toprule
\textbf{Coordenada y} & \textbf{Valor y} \\
\midrule
\endfirsthead

\toprule
\textbf{Coordenada y} & \textbf{Valor y} \\
\midrule
\endhead

\midrule \multicolumn{2}{r}{{Continua na próxima página}} \\ \midrule
\endfoot

\bottomrule
\endlastfoot
1 & -3310.000000 \\
2 & 775.223643 \\
3 & -59.101848 \\
4 & -763.889280 \\
5 & -3310.000000 \\
6 & -765.483076 \\
7 & -0.463918 \\
8 & -0.000000 \\
9 & -0.670103 \\
10 & -0.000000 \\
11 & -0.000000 \\
12 & -6.206186 \\
13 & -0.000000 \\
14 & -765.483076 \\
15 & -14.412371 \\
16 & -0.288660 \\
17 & -19.701031 \\
18 & -36.340206 \\
19 & -0.000000 \\
20 & -455.063711 \\
21 & -11.598768 \\
22 & -0.000000 \\
23 & -659.273969 \\
24 & -13.000000 \\
25 & -857.261769 \\
26 & -0.000000 \\
27 & -38.070780 \\
28 & -220.618050 \\
29 & -1834.516924 \\
30 & -1124.516924 \\
31 & -446.463918 \\
32 & -446.463918 \\
33 & -446.463918 \\
34 & -781.411231 \\
35 & -928.097833 \\
36 & 748.144542 \\
37 & -0.000000 \\
38 & -137.516924 \\
39 & -540.255448 \\
40 & 764.014273 \\
41 & -889.733764 \\
42 & -509.350515 \\
43 & -509.350515 \\
44 & -509.350515 \\
45 & -826.746560 \\
46 & -780.266236 \\
47 & -765.483076 \\
48 & -452.516924 \\
49 & -0.000000 \\
50 & -359.108319 \\
51 & 2.005833 \\
52 & -0.000000 \\
53 & -0.000000 \\
54 & -105.593997 \\
55 & -765.483076 \\
56 & -0.000000 \\

\end{longtable}


\section{Observações}

\begin{itemize}
\item O solver HiGHS foi configurado com o método IPM (Interior Point Method).
\item Este arquivo contém a solução detalhada para o problema Adlittle.
\item As variáveis duais representam os multiplicadores de Lagrange das restrições.
\item Os custos reduzidos (z) indicam o impacto de forçar variáveis não-básicas na base.
\end{itemize}

\end{document}
