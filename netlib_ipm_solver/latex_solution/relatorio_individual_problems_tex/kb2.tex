
\documentclass[12pt]{article}
\usepackage[utf8]{inputenc}
\usepackage[portuguese]{babel}
\usepackage{booktabs}
\usepackage{array}
\usepackage{geometry}
\usepackage{amsmath}
\usepackage{amsfonts}
\usepackage{longtable}

\geometry{a4paper, margin=1.5cm}

\title{Solução do Problema kb2 - Solver HiGHS}
\author{Análise Computacional}
\date{\today}

\begin{document}

\maketitle

\section{Informações do Problema}

\textbf{Informações do Problema:}
\begin{itemize}
\item Nome: kb2
\item Número de Variáveis: 41
\item Número de Restrições: 43
\item Inviabilidade Primal: 0.000e+00
\item Inviabilidade Dual: 0.000e+00
\item Valor Primal: -1.750e+03
\item Valor Dual: 0.000e+00
\item Gap: -7.423e-20
\item Número de Iterações: 18
\end{itemize}


\section{Variáveis Primais e Custos Reduzidos}

\begin{longtable}{@{}cccc@{}}
\caption{Variáveis primais e custos reduzidos do problema kb2} \\
\toprule
\textbf{Coordenada x} & \textbf{Valor x} & \textbf{Coordenada z} & \textbf{Valor z} \\
\midrule
\endfirsthead

\toprule
\textbf{Coordenada x} & \textbf{Valor x} & \textbf{Coordenada z} & \textbf{Valor z} \\
\midrule
\endhead

\midrule \multicolumn{4}{r}{{Continua na próxima página}} \\ \midrule
\endfoot

\bottomrule
\endlastfoot
1 & 0.811824 & 1 & 0.000000 \\
2 & 0.000000 & 2 & 0.063812 \\
3 & 0.000000 & 3 & 0.048962 \\
4 & 0.000000 & 4 & 0.102083 \\
5 & 4.672552 & 5 & 0.000000 \\
6 & 25.061124 & 6 & 0.000000 \\
7 & 5.000000 & 7 & 0.000000 \\
8 & 0.000000 & 8 & 0.046433 \\
9 & 0.000000 & 9 & 0.033953 \\
10 & 9.550968 & 10 & 0.000000 \\
11 & 0.000000 & 11 & 0.088802 \\
12 & 2.506552 & 12 & 0.000000 \\
13 & 8.779497 & 13 & 0.000000 \\
14 & 0.000000 & 14 & 0.003517 \\
15 & 9.188176 & 15 & 0.000000 \\
16 & 20.000000 & 16 & 0.000000 \\
17 & 15.449032 & 17 & 0.000000 \\
18 & 12.000000 & 18 & 0.000000 \\
19 & 8.391585 & 19 & 0.000000 \\
20 & 1.159379 & 20 & 0.000000 \\
21 & 0.000000 & 21 & 0.016343 \\
22 & 122.570690 & 22 & 0.000000 \\
23 & 10.000000 & 23 & -17.269208 \\
24 & 20.000000 & 24 & -17.214102 \\
25 & 25.000000 & 25 & -16.942046 \\
26 & 12.000000 & 26 & -16.659838 \\
27 & 15.570690 & 27 & 0.000000 \\
28 & 35.000000 & 28 & -17.209369 \\
29 & 5.000000 & 29 & -1.425779 \\
30 & 122.570690 & 30 & 0.000000 \\
31 & 15.050888 & 31 & 0.000000 \\
32 & 35.545500 & 32 & 0.000000 \\
33 & 20.837017 & 33 & 0.000000 \\
34 & 49.674172 & 34 & 0.000000 \\
35 & 66.188172 & 35 & 0.000000 \\
36 & 3214.889184 & 36 & 0.000000 \\
37 & 3597.519648 & 37 & 0.000000 \\
38 & 1770.361014 & 38 & 0.000000 \\
39 & 2009.742955 & 39 & 0.000000 \\
40 & 5651.993150 & 40 & 0.000000 \\
41 & 6262.646874 & 41 & 0.000000 \\

\end{longtable}

\section{Variáveis Duais (Multiplicadores de Lagrange)}

\begin{longtable}{@{}cc@{}}
\caption{Variáveis duais do problema kb2} \\
\toprule
\textbf{Coordenada y} & \textbf{Valor y} \\
\midrule
\endfirsthead

\toprule
\textbf{Coordenada y} & \textbf{Valor y} \\
\midrule
\endhead

\midrule \multicolumn{2}{r}{{Continua na próxima página}} \\ \midrule
\endfoot

\bottomrule
\endlastfoot
1 & 17.269208 \\
2 & 17.214102 \\
3 & 16.942046 \\
4 & 16.659838 \\
5 & 12.000000 \\
6 & 17.209369 \\
7 & 17.425779 \\
8 & 16.593471 \\
9 & 16.460185 \\
10 & 16.462337 \\
11 & 16.500000 \\
12 & 14.490664 \\
13 & 15.040967 \\
14 & 15.148758 \\
15 & -0.000000 \\
16 & 0.020368 \\
17 & -0.000000 \\
18 & -0.000000 \\
19 & -0.000000 \\
20 & 0.011439 \\
21 & -0.000000 \\
22 & 0.011724 \\
23 & 0.006393 \\
24 & -0.000000 \\
25 & -0.000000 \\
26 & 0.006934 \\
27 & 0.027901 \\
28 & 0.024157 \\
29 & 0.022367 \\
30 & 0.020292 \\
31 & 0.006522 \\
32 & -0.000000 \\
33 & -0.000000 \\
34 & -0.000000 \\
35 & -0.079006 \\
36 & -0.000000 \\
37 & -0.000000 \\
38 & -0.077823 \\
39 & -0.000000 \\
40 & -0.000000 \\
41 & -0.000000 \\
42 & -0.079267 \\
43 & -0.000000 \\

\end{longtable}


\section{Observações}

\begin{itemize}
\item O solver HiGHS foi configurado com o método IPM (Interior Point Method).
\item Este arquivo contém a solução detalhada para o problema kb2.
\item As variáveis duais representam os multiplicadores de Lagrange das restrições.
\item Os custos reduzidos (z) indicam o impacto de forçar variáveis não-básicas na base.
\end{itemize}

\end{document}
