
\documentclass[12pt]{article}
\usepackage[utf8]{inputenc}
\usepackage[portuguese]{babel}
\usepackage{booktabs}
\usepackage{array}
\usepackage{geometry}
\usepackage{amsmath}
\usepackage{amsfonts}
\usepackage{longtable}

\geometry{a4paper, margin=1.5cm}

\title{Solução do Problema DFL001 - Solver HiGHS}
\author{Análise Computacional}
\date{\today}

\begin{document}

\maketitle

\section{Informações do Problema}

\textbf{Informações do Problema:}
\begin{itemize}
\item Nome: DFL001
\item Número de Variáveis: 12230
\item Número de Restrições: 6071
\item Inviabilidade Primal: N/A
\item Inviabilidade Dual: N/A
\item Valor Primal: N/A
\item Valor Dual: N/A
\item Gap: N/A
\item Número de Iterações: 0
\end{itemize}


\section{Status}

\textbf{Status:} Infeasible - Problema não convergiu.


\section{Observações}

\begin{itemize}
\item O solver HiGHS foi configurado com o método IPM (Interior Point Method).
\item Este arquivo contém a solução detalhada para o problema DFL001.
\item As variáveis duais representam os multiplicadores de Lagrange das restrições.
\item Os custos reduzidos (z) indicam o impacto de forçar variáveis não-básicas na base.
\end{itemize}

\end{document}
