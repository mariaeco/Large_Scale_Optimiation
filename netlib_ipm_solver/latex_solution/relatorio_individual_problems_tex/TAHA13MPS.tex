
\documentclass[12pt]{article}
\usepackage[utf8]{inputenc}
\usepackage[portuguese]{babel}
\usepackage{booktabs}
\usepackage{array}
\usepackage{geometry}
\usepackage{amsmath}
\usepackage{amsfonts}
\usepackage{longtable}

\geometry{a4paper, margin=1.5cm}

\title{Solução do Problema TAHA13MPS - Solver HiGHS}
\author{Análise Computacional}
\date{\today}

\begin{document}

\maketitle

\section{Informações do Problema}

\textbf{Informações do Problema:}
\begin{itemize}
\item Nome: TAHA13MPS
\item Número de Variáveis: 2
\item Número de Restrições: 4
\item Inviabilidade Primal: 0.000e+00
\item Inviabilidade Dual: 0.000e+00
\item Valor Primal: -2.100e+01
\item Valor Dual: -2.100e+01
\item Gap: -0.000e+00
\item Número de Iterações: 4
\end{itemize}


\section{Variáveis Primais e Custos Reduzidos}

\begin{longtable}{@{}cccc@{}}
\caption{Variáveis primais e custos reduzidos do problema TAHA13MPS} \\
\toprule
\textbf{Coordenada x} & \textbf{Valor x} & \textbf{Coordenada z} & \textbf{Valor z} \\
\midrule
\endfirsthead

\toprule
\textbf{Coordenada x} & \textbf{Valor x} & \textbf{Coordenada z} & \textbf{Valor z} \\
\midrule
\endhead

\midrule \multicolumn{4}{r}{{Continua na próxima página}} \\ \midrule
\endfoot

\bottomrule
\endlastfoot
1 & 3.000000 & 1 & 0.000000 \\
2 & 1.500000 & 2 & 0.000000 \\

\end{longtable}

\section{Variáveis Duais (Multiplicadores de Lagrange)}

\begin{longtable}{@{}cc@{}}
\caption{Variáveis duais do problema TAHA13MPS} \\
\toprule
\textbf{Coordenada y} & \textbf{Valor y} \\
\midrule
\endfirsthead

\toprule
\textbf{Coordenada y} & \textbf{Valor y} \\
\midrule
\endhead

\midrule \multicolumn{2}{r}{{Continua na próxima página}} \\ \midrule
\endfoot

\bottomrule
\endlastfoot
1 & -0.750000 \\
2 & -0.500000 \\
3 & -0.000000 \\
4 & -0.000000 \\

\end{longtable}


\section{Observações}

\begin{itemize}
\item O solver HiGHS foi configurado com o método IPM (Interior Point Method).
\item Este arquivo contém a solução detalhada para o problema TAHA13MPS.
\item As variáveis duais representam os multiplicadores de Lagrange das restrições.
\item Os custos reduzidos (z) indicam o impacto de forçar variáveis não-básicas na base.
\end{itemize}

\end{document}
